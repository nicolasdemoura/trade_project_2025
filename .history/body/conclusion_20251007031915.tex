\section{Conclusion}

This study applies the \cite{costinot2012TheReviewofEconomicStudies} multi-sector Ricardian trade model to examine U.S. tariff policies using WIOD input-output data and U.S. tariff information. The analysis covers 10 countries and 12 sectors, revealing asymmetric welfare effects under the 2025 YTD tariff scenario. With mobile labor, the model predicts welfare gains of 19.58\% for the European Union and 14.37\% for the United States, while showing losses of 4.91\% for Mexico and 1.28\% for Canada. These results suggest that trade policies redistribute welfare across countries, with larger economies potentially benefiting at the expense of smaller, trade-dependent partners.

The calibration strategy combines input-output data with trade policy information, achieving reasonable fit to bilateral trade patterns ($R^2 = 0.933$). However, the model struggles to match GDP shares simultaneously, indicating trade-offs between trade flow accuracy and production pattern replication. The immobile labor specification produces implausibly large welfare changes, suggesting numerical issues and highlighting the importance of labor mobility assumptions in these models.

The results indicate that the examined tariff policies function more as redistributive mechanisms than efficiency improvements. While the welfare effects are substantial, ranging from losses of nearly 5\% to gains exceeding 19\%, these findings should be interpreted cautiously given the model's limitations. The contrast between mobile and immobile labor results emphasizes the role of labor market assumptions in determining policy impacts. Future work could address computational challenges in solving constrained equilibrium models and estimate more precisely what are the policy shocks driving these welfare changes.