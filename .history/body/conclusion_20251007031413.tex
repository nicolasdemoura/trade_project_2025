\section{Conclusion}

This study provides a quantitative assessment of contemporary U.S. tariff policies using structural estimation of the \cite{costinot2012TheReviewofEconomicStudies} multi-sector Ricardian trade model. By integrating WIOD input-output data with high-frequency U.S. tariff information, we deliver precise welfare measurements that account for general equilibrium adjustments across 10 countries and 12 sectors. Our analysis reveals significant asymmetric welfare effects, with the 2025 YTD tariff scenario generating welfare gains of 19.58\% for the European Union and 14.37\% for the United States under mobile labor specification, while imposing welfare losses of 4.91\% on Mexico and 1.28\% on Canada. These results demonstrate that trade policies create substantial redistributive effects across countries, with large diversified economies benefiting at the expense of smaller, trade-dependent partners through both direct protection benefits and indirect trade diversion mechanisms.

Our four-stage calibration strategy demonstrates the feasibility of combining input-output data with high-frequency trade policy information for quantitative trade analysis, achieving excellent fit to observed bilateral trade patterns ($R^2 = 0.933$). However, the model's inability to simultaneously match GDP shares highlights tensions between trade flow accuracy and production pattern replication in structural trade models. The comparison between mobile and immobile labor scenarios reveals critical limitations, as the immobile specification produces implausibly large gains exceeding 14,000\% that suggest numerical instability rather than meaningful economic relationships. This underscores the importance of labor mobility assumptions and indicates areas for future methodological development, including improved computational methods for solving general equilibrium models under binding constraints and incorporation of non-tariff measures and dynamic adjustment mechanisms.

The welfare analysis provides important insights for trade policy evaluation, revealing that contemporary policies function as redistributive mechanisms across countries rather than efficiency-enhancing reforms, with effects ranging from losses of nearly 5\% to gains exceeding 19\%. The substantial magnitude of these welfare effects justifies careful quantitative evaluation in policy design, while the contrast between mobile and immobile labor results highlights the critical importance of labor market flexibility in determining adjustment costs. This study demonstrates that structural trade models can provide valuable quantitative guidance for policy evaluation when properly calibrated to high-quality data, and the significant welfare effects identified underscore the importance of considering general equilibrium impacts as countries continue to adjust their trade policies in response to changing global conditions.
