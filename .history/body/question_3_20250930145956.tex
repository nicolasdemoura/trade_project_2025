\begin{solution}
	extbf{Restatement.} Provide intuition for how opening to trade affects each closed economy, and explain why low-productivity firms exit while resources reallocate toward high-productivity firms within industries (slides’ Melitz mechanisms).

\bigskip
	extbf{Key forces from the slides.}
\begin{itemize}
	\item \emph{Love-of-variety and CES demand:} Consumers value variety and substitute toward lower-price (more efficient) varieties with elasticity $\sigma>1$.
	\item \emph{Monopolistic competition with constant markup:} Each firm charges $p(\varphi)=\tfrac{\sigma}{\sigma-1}\tfrac{1}{\varphi}$ with $w=1$.
	\item \emph{Fixed operating/export costs:} Serving a market entails fixed costs ($f$ domestically; $f_x$ for exporting) and potentially iceberg variable trade costs ($\tau>1$).
\end{itemize}

\bigskip
	extbf{What trade changes relative to autarky.}
\begin{enumerate}
	\item \emph{Tougher competition (market expansion):} Opening to trade expands the set of available varieties in each country. For a given domestic firm, the relevant demand becomes more elastic at the firm’s price because consumers can substitute to imported varieties. This intensifies competition.
	\item \emph{Selection (exit of the least productive):} With tougher competition, a higher productivity is needed to cover the domestic fixed cost $f$. In notation, the domestic cutoff $\varphi^*$ rises. Firms with $\varphi<\varphi^*$ exit because their revenue $r(\varphi)$ can no longer cover $f$ at the constant markup. This is the \emph{selection effect} emphasized in the slides.
	\item \emph{Export market access for the most productive:} High-productivity firms ($\varphi\ge \varphi_x^*>\varphi^*$) can profitably pay the export fixed cost $f_x$ and iceberg cost $\tau$, generating additional revenue from abroad. This expands their scale.
	\item \emph{Reallocation within industries:} Labor (and market shares) shift from exiting/less productive firms to surviving/more productive firms. The average productivity among active producers rises because the lower tail exits and the upper tail expands output (and exports). This \emph{within-industry reallocation} is a central source of gains in the slides.
	\item \emph{Price-index reduction and welfare gains:} The increase in the mass of available varieties (home + imported) and the shift of expenditure toward lower-price (higher-$\varphi$) varieties lowers the CES price index $P$. With $U=R/P$ in the CES setting (for given $R$), this yields a real-income gain.
\end{enumerate}

\bigskip
	extbf{Economic intuition behind exit and reallocation.}
\begin{itemize}
	\item \emph{Exit of low-$\varphi$ firms:} When imports arrive, consumers can buy close substitutes at prices set by more efficient foreign producers. Given constant markups, unit prices reflect marginal costs; low-$\varphi$ firms have high marginal costs and thus high prices, so their quantity and revenue fall below the threshold needed to cover $f$. They exit.
	\item \emph{Resource reallocation:} Labor freed by exiting firms is absorbed by surviving high-$\varphi$ firms that expand to serve both domestic and foreign demand. Because these firms transform labor more efficiently into output, reallocating labor to them raises aggregate productivity.
	\item \emph{Cutoff ranking:} With trade costs ($\tau>1$, $f_x>0$), the export cutoff satisfies $\varphi_x^*>\varphi^*$, i.e., only the top producers export. This magnifies the scale of the most productive firms and the fall in $P$.
\end{itemize}

\bigskip
	extbf{Bottom line (slides’ message).}
Opening to trade induces: (i) exit of the least productive firms (higher $\varphi^*$), (ii) expansion of the most productive firms (some start exporting if $\varphi\ge\varphi_x^*$), (iii) reallocation of resources toward more productive producers, and (iv) lower $P$ and higher real income. These within-industry effects are present even between symmetric countries.

\medskip
\noindent\boxed{\text{Trade causes selection and reallocation: low-$\varphi$ exits, high-$\varphi$ expands (often exports), }P\downarrow,\;\text{real income}\;\uparrow.}
\end{solution}
