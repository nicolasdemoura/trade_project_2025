Solve for the closed economy values of $\varphi^*$, $P$, $R$.

\begin{solution}
    \paragraph{Household side} The household's problem is given by:
    \begin{align*}
        \max_{q(\omega)} \quad & U = \left( \int_{\omega \in \Omega} q(\omega)^{\rho} d\omega \right)^{\frac{1}{\rho}} \\
        \text{s.t.} \quad & \int_{\omega \in \Omega} p(\omega) q(\omega) d\omega = R
    \end{align*}
    where $\rho = \frac{\sigma - 1}{\sigma}$ and $\sigma > 1$. The Lagrangian for this problem is:
    \begin{align*}
        \mathcal{L} = \left( \int_{\omega \in \Omega} q(\omega)^{\rho} d\omega \right)^{\frac{1}{\rho}} + \lambda \left( R - \int_{\omega \in \Omega} p(\omega) q(\omega) d\omega \right)
    \end{align*}
    The first order condition with respect to $q(\omega)$ is:
    \begin{align*}
        \frac{\partial \mathcal{L}}{\partial q(\omega)} = 0 &\iff \frac{1}{\rho} \left( \int_{\omega \in \Omega} q(\omega)^{\rho} d\omega \right)^{\frac{1}{\rho} - 1} \rho q(\omega)^{\rho - 1} - \lambda p(\omega) = 0 \\
        &\iff q(\omega)^{\rho - 1} = \lambda p(\omega) \left( \int_{\omega \in \Omega} q(\omega)^{\rho} d\omega \right)^{1 - \frac{1}{\rho}} \\
        &\iff q(\omega) = \left[ \lambda p(\omega) \right]^{\frac{1}{\rho - 1}} \left( \int_{\omega \in \Omega} q(\omega)^{\rho} d\omega \right)^{\frac{1}{\rho} } \\
        &\iff q(\omega) = \left[ \lambda p(\omega) \right]^{-\sigma} Q 
    \end{align*}
    where $Q = \left( \int_{\omega \in \Omega} q(\omega)^{\rho} d\omega \right)^{\frac{1}{\rho}}$ is the CES aggregate. Substituting this expression for $q(\omega)$ into the budget constraint, we have:
    \begin{align*}
        R &= \int_{\omega \in \Omega} p(\omega) q(\omega) d\omega \\
        &= \int_{\omega \in \Omega} p(\omega) \left[ \lambda p(\omega) \right]^{-\sigma} Q d\omega \\
        &= Q \lambda^{-\sigma} \int_{\omega \in \Omega} p(\omega)^{1 - \sigma} d\omega \\
        &\iff \lambda^{-\sigma} = \frac{R}{Q P^{1 - \sigma}} 
    \end{align*}
    where $P = \left( \int_{\omega \in \Omega} p(\omega)^{1 - \sigma} d\omega \right)^{\frac{1}{1 - \sigma}}$ is the CES price index. Substituting this expression for $\lambda^{-\sigma}$ back into the expression for $q(\omega)$, we have:
    \begin{align*}
        q(\omega) &= \left[ \frac{R}{Q P^{1 - \sigma}} p(\omega) \right]^{-\sigma} Q \\
        &\iff q(\omega) = \frac{R}{P} \left( \frac{p(\omega)}{P} \right)^{-\sigma}\\
        &\iff q(\omega) = Q \left( \frac{p(\omega)}{P} \right)^{-\sigma}
    \end{align*}
    The revenue associated with a variety $\omega$ is:
    \begin{align*}
        r(\omega) &= p(\omega) q(\omega) \\
        &= p(\omega) \frac{R}{P} \left( \frac{p(\omega)}{P} \right)^{-\sigma} \\
        &\iff r(\omega) = R \left( \frac{p(\omega)}{P} \right)^{1 - \sigma}
    \end{align*}
    
    \paragraph{Producer side.} A firm with productivity $\varphi$ needs $\dfrac{1}{\varphi}$ units of labor to produce one unit of output. It also pays a fixed cost $f$ units of labor to operate. Hence, the total cost of producing quantity $q$ is:
    \begin{align*}
        C(q, \varphi) = f + \frac{q}{\varphi} 
    \end{align*}
    
    The profit maximization problem for the firm is:
    \begin{align*}
        \max_{p(\varphi)} \quad & \pi(\varphi) = r(\varphi) - w\left( f + \frac{q(\varphi)}{\varphi} \right)
    \end{align*}
    where $w$ is the wage. Normalizing the wage to $w = 1$:
    \begin{align*}
        \max_{p(\varphi)} \quad & \pi(\varphi) = r(\varphi) - \left( f + \frac{q(\varphi)}{\varphi} \right)
    \end{align*}
    The FOC with respect to $p(\varphi)$ is:
    \begin{align*}
        \frac{\partial \pi(\varphi)}{\partial p(\varphi)} = 0 &\iff \frac{\partial r(\varphi)}{\partial p(\varphi)} - \frac{1}{\varphi} \frac{\partial q(\varphi)}{\partial p(\varphi)} = 0 \\
        &\iff (1 - \sigma) r(\varphi) + \frac{\sigma}{\varphi} q(\varphi) = 0 \\
        &\iff r(\varphi) = \frac{\sigma}{\sigma - 1} \frac{1}{\varphi} q(\varphi) \\
        &\iff p(\varphi) = \frac{\sigma}{\sigma - 1} \frac{1}{\varphi} = \frac{1}{\rho\varphi} 
    \end{align*}
    Thus, the optimal price is a constant markup over marginal cost.
    
    Substituting into the revenue expression and using demand,
    \begin{align*}
        r(\varphi) &= R \left( \frac{p(\varphi)}{P} \right)^{1 - \sigma} \\
        &= R \left( \frac{1}{\rho \varphi P} \right)^{1 - \sigma} \\
        &= (\rho \varphi)^{\sigma - 1} R P^{\sigma - 1}
    \end{align*} 
    Profits from serving the domestic market are: 
    \begin{align*}
        \pi(\varphi) &= r(\varphi) - \left( f + \frac{q(\varphi)}{\varphi} \right) \\
        &= r(\varphi) - f - \frac{1}{\varphi} Q \left( \frac{p(\varphi)}{P} \right)^{-\sigma} \\
        &= r(\varphi) - f - \frac{1}{\varphi} Q \left( \frac{1}{\rho\varphi P} \right)^{-\sigma} \\
        &= r(\varphi) - f -  \rho^{\sigma} \varphi^{\sigma - 1} R P^{\sigma - 1} \\
        &= r(\varphi) - f -  \rho r(\varphi) \\
        &= \frac{1}{\sigma} r(\varphi) - f
    \end{align*}
    
    The marginal (least productive) active firm satisfies $\pi(\varphi^*)=0$, hence:
    \begin{align*}
        \pi(\varphi^*) = 0 &\iff \frac{1}{\sigma} r(\varphi^*) - f = 0 \\
        &\iff r(\varphi^*) = \sigma f 
    \end{align*}
    Note that:
    \begin{align*}
        r(\varphi) = \left(\frac{\varphi}{\varphi^*}\right)^{\sigma - 1} \sigma f
    \end{align*}

    \paragraph{Enter and Exit.} Firms pay a sunk entry cost $f_e$ to draw a productivity $\varphi$ from the distribution $G$. If they operate, they pay the fixed cost $f$ and produce optimally. If not, they exit and earn zero. The probability of drawing $\varphi \ge \varphi^*$ is $1 - G(\varphi^*)$. The ex-post productivity distribution among active firms is:
    \begin{align*}
        \mu(\varphi) = \begin{cases}
            \frac{g(\varphi)}{1 - G(\varphi^*)} & \text{if } \varphi \geq \varphi^*\\
            0 & \text{otherwise}
        \end{cases}
    \end{align*}
    where $g(\varphi)$ is the density associated with $G$.

    After entry, there is a probability $\delta$ that the firm exits exogenously. Hence, the value of a firm with productivity $\varphi$ is:
    \begin{align*}
        v(\varphi) = \max\left\{\frac{\pi(\varphi)}{\delta}, 0\right\}
    \end{align*}
    The free entry condition requires the expected value of entry to equal the entry cost:
    \begin{align*}
        \int v(\varphi) g(\varphi) d\varphi = f_e &\iff (1-G(\varphi^*))\int_{\varphi^*}^{\infty} \frac{\pi(\varphi)}{\delta} \frac{g(\varphi)}{1 - G(\varphi^*)} d\varphi = f_e \\
        &\iff \frac{1-G(\varphi^*)}{\delta} \bar{\pi} = f_e
    \end{align*}
    where $\bar{\pi} = \int_{\varphi^*}^{\infty} \pi(\varphi) \frac{g(\varphi)}{1 - G(\varphi^*)} d\varphi$ is the average profit among active firms.

    \paragraph{Pareto Distribution and shorthand.} Assume the productivity distribution is Pareto with parameters $\varphi_{min}$ and $k > \sigma - 1$:
    \begin{align*}
        G(\varphi) = 1 - \left( \frac{\varphi_{min}}{\varphi} \right)^k, \quad g(\varphi) = k \frac{\varphi_{min}^k}{\varphi^{k+1}}, \quad \text{for } \varphi \geq \varphi_{min}
    \end{align*}
    It is convenient to define the frequently appearing fraction
    \begin{align*}
        \theta \equiv \frac{k}{k+1-\sigma}.
    \end{align*}
    Then:
    \begin{align*}
        \bar{\pi} &= \int_{\varphi^*}^{\infty} \left( \frac{1}{\sigma} r(\varphi) - f \right) \frac{g(\varphi)}{1 - G(\varphi^*)} d\varphi \\
        &= \frac{fk}{k + 1 - \sigma} \;=\; f\,\theta
    \end{align*}
    where the last step uses $k > \sigma - 1$. Substituting this into free entry and using $1-G(\varphi^*)=(\varphi_{min}/\varphi^*)^k$:
    \begin{align*}
        \left(\frac{\varphi_{min}}{\varphi^*}\right)^k = \frac{\delta f_e}{\bar{\pi}} = \frac{\delta f_e}{f\theta}
        \quad\implies\quad
        \varphi^* = \varphi_{min}\left(\frac{f\theta}{\delta f_e}\right)^{\frac{1}{k}}.
    \end{align*}
    Thus, the cutoff $\varphi^*$ is in closed form. 

    \paragraph{Weighted Average Productivity.} The weighted average productivity of active firms is:
    \begin{align*}
        \tilde{\varphi} &= \left( \int_{\varphi^*}^{\infty} \varphi^{\sigma - 1} \frac{g(\varphi)}{1 - G(\varphi^*)} d\varphi \right)^{\frac{1}{\sigma - 1}}
        = \left( \frac{k}{k + 1 - \sigma} \right)^{\frac{1}{\sigma - 1}} \varphi^*
        = \theta^{\frac{1}{\sigma-1}}\varphi^* .
    \end{align*}

    \paragraph{Aggregate Price Index.} The price index is:
    \begin{align*}
        P &= \left( \int_{\varphi^*}^{\infty} p(\varphi)^{1 - \sigma} \, M \frac{g(\varphi)}{1 - G(\varphi^*)} d\varphi \right)^{\frac{1}{1 - \sigma}} 
        = M^{\frac{1}{1 - \sigma}} \frac{1}{\rho \tilde{\varphi}}
        = M^{\frac{1}{1 - \sigma}} \frac{1}{\rho} \theta^{-\frac{1}{\sigma-1}}\frac{1}{\varphi^*}.
    \end{align*}

    \paragraph{Aggregate Revenue and mass of firms.} In steady state, $[1-G(\varphi^*)] M_e = \delta M$. Using free entry, $M_e f_e = M\bar{\pi}$, so $[1-G(\varphi^*)] = \delta f_e/\bar{\pi}$ (already used above). Labor market clearing implies $R=L$. Average revenue and $R=M\bar{r}$ give:
    \begin{align*}
        R &= M \int_{\varphi^*}^{\infty} r(\varphi) \frac{g(\varphi)}{1 - G(\varphi^*)} d\varphi 
        = M \sigma f \left(\frac{\tilde{\varphi}}{\varphi^*}\right)^{\sigma - 1}
        = M \sigma f \theta,
    \end{align*}
    hence, since $R=L$,
    \begin{align*}
        M = \frac{L}{\sigma f \theta} = \frac{L}{\sigma f}\,\theta^{-1}.
    \end{align*}

    \paragraph{Closed-form $P$ and simplification.} Plugging $M$ and $\tilde{\varphi}$ into $P$,
    \begin{align*}
        P &= M^{\frac{1}{1-\sigma}} \frac{1}{\rho \tilde{\varphi}}
        = \frac{1}{\rho}\left(\frac{L}{\sigma f}\,\theta^{-1}\right)^{\frac{1}{1-\sigma}}
           \left(\theta^{\frac{1}{\sigma-1}}\varphi^*\right)^{-1} \\
          &= \frac{1}{\rho}\left(\frac{L}{\sigma f}\right)^{\frac{1}{1-\sigma}}\frac{1}{\varphi^*}
    \end{align*}
    Using the cutoff,
    \begin{align*}
        \frac{1}{\varphi^*} = \frac{1}{\varphi_{min}}\left(\frac{\delta f_e}{f\theta}\right)^{\frac{1}{k}}
        = \frac{1}{\varphi_{min}}\,(\delta f_e)^{\frac{1}{k}} f^{-\frac{1}{k}} \theta^{-\frac{1}{k}},
    \end{align*}
    we obtain the general closed form
    \begin{align*}
        P &= \frac{1}{\rho}\left(\frac{\sigma f}{L}\right)^{\frac{1}{\sigma-1}}
             \frac{1}{\varphi_{min}}\,
             (\delta f_e)^{\frac{1}{k}} f^{-\frac{1}{k}} \theta^{-\frac{1}{k}} \\[2pt]
          &= \frac{\sigma^{\frac{1}{\sigma-1}}}{\rho}\;
             f^{\frac{1}{\sigma-1}-\frac{1}{k}}\;
             (\delta f_e)^{\frac{1}{k}}\;
             \theta^{-\frac{1}{k}}\;
             L^{-\frac{1}{\sigma-1}}\;
             \varphi_{min}^{-1},
             \qquad \theta=\frac{k}{k+1-\sigma}.
    \end{align*}
\end{solution}
