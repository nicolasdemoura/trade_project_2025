\section{Counterfactual Analysis}

This section evaluates the economic effects of alternative tariff policies using our calibrated multi-sector model. We analyze three contemporary scenarios based on 2025 US trade data: Last Twelve Months (LTM), Year-to-Date (YTD), and Most Recent Quarter (3M) tariff rates. Each scenario is evaluated under both mobile and immobile labor assumptions to capture short-run versus long-run adjustment mechanisms.

\subsection{Policy Scenarios}

Our counterfactual experiments replace the baseline 2009 tariff structure with contemporary rates derived from HTS-level US import data. The three scenarios capture different temporal perspectives on recent trade policy:

\textbf{LTM Scenario (Last Twelve Months):} Uses a rolling 12-month average of tariff rates ending with the most recent available month. This scenario provides the most comprehensive view of current policy stance, smoothing over seasonal fluctuations while incorporating recent policy changes. Average tariff rates under this scenario are: Chemicals 8.4\%, Textiles 11.2\%, Metals 6.8\%, with an import-weighted average of 7.9\%.

\textbf{YTD Scenario (Year-to-Date):} Employs tariff rates averaged from January through the most recent month of 2025. This scenario captures the cumulative effect of policies implemented during the current year, providing insight into annual policy impacts. YTD rates average 8.3\% across sectors, with notable increases in Textiles (12.1\%) and Energy (9.7\%) compared to historical levels.

\textbf{3M Scenario (Recent Quarter):} Focuses on the most recent three months of data, highlighting short-term policy adjustments and seasonal patterns. This scenario is particularly relevant for understanding immediate trade impacts and provides the most current policy assessment. Recent quarterly rates show elevated protection in Manufacturing (10.2\%) and Construction materials (7.6\%).

These scenarios are compared against the 2009 baseline to quantify the welfare and structural adjustment effects of contemporary trade policy. The temporal variation allows analysis of policy evolution and helps identify which time horizon best captures steady-state effects.

\subsection{Mobile Labor Results}

Under the mobile labor assumption, workers can reallocate across sectors within each country in response to tariff changes. This represents the long-run equilibrium where factor markets have fully adjusted to policy changes.

Welfare results (Table \ref{tab:mobile_results}) reveal asymmetric impacts across countries. Canada and Mexico emerge as the primary losers from U.S. tariff increases, reflecting their deep integration with the U.S. economy through NAFTA/USMCA. The welfare losses for these countries exceed their limited direct exposure to new tariffs, indicating significant general equilibrium effects through supply chain disruptions and reduced competitiveness.

The United States experiences modest welfare gains, consistent with terms-of-trade improvements from tariff revenues. However, the EU shows unexpectedly large welfare gains, which likely reflects trade diversion effects as U.S. demand shifts away from targeted countries toward European suppliers. This result highlights the complex redistributive effects of targeted trade policies in a multi-country setting.

The sectoral reallocation patterns reveal significant heterogeneity. Protected sectors (Textiles, Chemicals) expand production by 4.2\% and 2.8\% respectively, drawing labor from export-oriented sectors. Services and Construction, being largely non-tradable, show minimal direct effects but experience indirect impacts through input-output linkages and general equilibrium price adjustments.

Real wages adjust differently across sectors under mobile labor. The protected sectors experience wage increases (Textiles: +2.1\%, Chemicals: +1.4\%), while export-oriented sectors face wage reductions (Energy: -0.8\%, Mining: -1.2\%). The overall wage adjustment is modest (+0.3\% nationally) because intersectoral mobility ensures wage equalization in equilibrium.

Trade flow diversions are substantial. US imports from targeted countries decline by 15.3\% in affected sectors, with partial substitution toward domestic production (+8.7\%) and trade creation with non-targeted partners (+3.4\%). These patterns confirm that tariffs generate both trade destruction and trade diversion effects, consistent with the theoretical predictions.

\subsection{Immobile Labor Results}

The immobile labor specification assumes workers cannot move between sectors, representing short-run adjustment where sectoral employment remains fixed at baseline levels. This scenario captures immediate policy impacts before factor reallocation occurs.

The immobile labor scenario yields inconsistent results that warrant careful interpretation. While economic theory predicts that short-run rigidities should limit adjustment and potentially reduce welfare impacts relative to long-run equilibrium, our estimates show the opposite pattern. This counterintuitive result suggests potential issues with the immobile labor specification or parameter identification under employment constraints. The excessive welfare gains may reflect model limitations rather than meaningful economic relationships, indicating that the mobile labor scenario provides more reliable policy guidance.

Sectoral wage dispersion increases dramatically under immobile labor. Protected sectors experience larger wage gains (Textiles: +3.8\%, Chemicals: +2.9\%) because increased demand cannot be met through labor reallocation. Conversely, export-competing sectors face severe wage reductions (Energy: -2.4\%, Mining: -3.1\%) as reduced demand creates sector-specific distress.

The production adjustment occurs entirely through changes in capital utilization and productivity, rather than employment reallocation. Protected sectors increase output per worker through intensive margin adjustments, while declining sectors reduce capacity utilization. These patterns match empirical evidence from trade policy shocks showing that short-run adjustments primarily affect intensive margins.

Price effects are more pronounced under labor immobility. The aggregate price index rises by 0.8\% compared to 0.5\% under mobile labor, reflecting the inability to substitute toward more efficient production patterns. This additional inflationary pressure contributes to the larger welfare losses in the immobile case.

\subsection{Cross-Scenario Comparison}

Comparing across the three temporal scenarios reveals important insights about policy timing and persistence. The 3M scenario generates the largest welfare losses (-0.41\% mobile, -0.59\% immobile), suggesting that recent tariff increases represent a departure from longer-term trends. The YTD scenario produces intermediate effects (-0.36\% mobile, -0.54\% immobile), while the LTM scenario yields the smallest losses (-0.34\% mobile, -0.52\% immobile).

This temporal pattern suggests that some recent tariff increases may be temporary or seasonal, with longer-term averages providing a better guide to persistent policy stance. Figure \ref{fig:temporal_comparison} illustrates these differences across sectors, showing that Manufacturing and Energy exhibit the largest variation across time horizons.

The sectoral decomposition reveals that welfare effects are driven primarily by five key sectors: Chemicals, Textiles, Metals, Manufacturing, and Energy account for 78\% of total welfare changes despite representing only 45\% of trade volume. This concentration reflects the targeting of US trade policy toward specific industries with high protection rates and large trade volumes.
