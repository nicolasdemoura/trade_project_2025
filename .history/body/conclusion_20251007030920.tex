\section{Conclusion}

This study provides a quantitative assessment of contemporary U.S. tariff policies using structural estimation of the \cite{costinot2012TheReviewofEconomicStudies} multi-sector Ricardian trade model. By integrating WIOD input-output data with high-frequency U.S. tariff information, we deliver precise welfare measurements that account for general equilibrium adjustments across 10 countries and 12 sectors.

\subsection{Key Findings}

Our analysis reveals significant asymmetric welfare effects from contemporary U.S. trade policies. Under the mobile labor specification, which represents long-run equilibrium adjustments, the 2025 YTD tariff scenario generates welfare gains of 19.58\% for the European Union and 14.37\% for the United States, while imposing welfare losses of 4.91\% on Mexico and 1.28\% on Canada. These results demonstrate that trade policies create substantial redistributive effects across countries, with large diversified economies benefiting at the expense of smaller, trade-dependent partners.

The pattern of welfare effects reflects both direct protection benefits and indirect trade diversion mechanisms. Countries experiencing large welfare gains appear to benefit from improved market access as global trade patterns shift away from targeted economies. The substantial losses faced by Mexico and Canada, despite their preferential trade status with the United States through USMCA, indicate that contemporary tariff policies generate effects beyond their direct sectoral coverage through input-output linkages and general equilibrium price adjustments.

\subsection{Methodological Contributions}

Our four-stage calibration strategy demonstrates the feasibility of combining input-output data with high-frequency trade policy information for quantitative trade analysis. The decomposed trade cost specification successfully reduces computational complexity while maintaining economic interpretability, achieving excellent fit to observed bilateral trade patterns ($R^2 = 0.933$). However, the model's inability to simultaneously match GDP shares highlights the tension between trade flow accuracy and production pattern replication in structural trade models.

The comparison between mobile and immobile labor scenarios reveals critical limitations in current modeling approaches. While mobile labor results yield economically interpretable welfare effects, the immobile labor specification produces implausibly large gains (exceeding 14,000\% for some countries) that suggest numerical instability rather than meaningful economic relationships. This finding underscores the importance of labor mobility assumptions in quantitative trade models and indicates areas for future methodological development.

\subsection{Policy Implications}

The welfare analysis provides several insights for trade policy evaluation. First, the asymmetric distribution of gains and losses suggests that contemporary trade policies function as redistributive mechanisms across countries rather than efficiency-enhancing reforms. Large economies with diversified trade portfolios can better absorb and benefit from policy changes, while smaller economies face disproportionate adjustment costs.

Second, the substantial welfare effects identified through our analysis indicate that trade policies generate economically meaningful impacts that justify careful quantitative evaluation. The magnitudes we estimate, ranging from losses of nearly 5\% to gains exceeding 19\%, represent significant economic effects that warrant consideration in policy design.

Third, the contrast between our mobile and immobile labor results highlights the critical importance of labor market flexibility in determining adjustment costs. Countries with more flexible labor markets may experience smoother adjustments to trade policy changes, while those with rigid labor markets could face severe dislocation costs.

\subsection{Limitations and Future Research}

Several limitations suggest directions for future research. The poor fit to GDP shares indicates that our model specification may be insufficiently flexible to capture all relevant aspects of production and trade patterns. Future work could explore alternative parameterizations or incorporate additional data sources to improve the model's ability to replicate observed economic structures.

The numerical instability in the immobile labor specification highlights the need for improved computational methods for solving general equilibrium models under binding constraints. Developing more stable solution algorithms could enable more reliable analysis of short-run adjustment mechanisms.

Our analysis focuses exclusively on tariff policies, abstracting from other trade barriers such as quotas, standards, or regulatory measures. Extending the framework to incorporate non-tariff measures could provide a more complete assessment of contemporary trade policy impacts.

Finally, our static framework cannot capture dynamic adjustment mechanisms or investment responses to policy changes. Incorporating capital accumulation and technology adoption could improve the realism of long-run policy assessments and provide insights into the intertemporal distribution of adjustment costs.

\subsection{Concluding Remarks}

This study demonstrates that structural trade models can provide valuable quantitative guidance for policy evaluation when properly calibrated to high-quality data. Our findings suggest that contemporary U.S. trade policies generate substantial welfare redistribution across countries, with effects that extend well beyond their direct sectoral coverage. While methodological challenges remain, particularly regarding labor mobility specifications and production pattern fitting, the framework provides a foundation for evidence-based trade policy analysis.

The significant welfare effects we identify underscore the importance of considering general equilibrium impacts in trade policy design. As countries continue to adjust their trade policies in response to changing global conditions, quantitative frameworks like the one developed here will become increasingly valuable for anticipating and managing the economic consequences of these policy changes.
