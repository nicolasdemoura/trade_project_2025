\section{Introduction}

\paragraph{} The resurgence of protectionist trade policies in recent years has reignited debates about the economic consequences of tariffs and their role in international trade. While classical trade theory suggests that tariffs generally reduce global welfare through distortions in comparative advantage, the practical implementation and distributional effects of these policies remain subjects of intense empirical investigation. Recent work by \cite{ignatenko2025making} highlights the complex mechanisms through which tariff policies affect economic outcomes, emphasizing the need for comprehensive quantitative frameworks to evaluate their multifaceted impacts.

\paragraph{} This research project investigates the long-term economic impacts of recent U.S. tariff policies on both domestic and global economies using the quantitative framework developed by \cite{costinot2012TheReviewofEconomicStudies}. The Costinot, Donaldson, and Komunjer (CDK) model provides an ideal analytical foundation for this investigation due to its multi-sector structure, which captures the complex interdependencies between sectors through input-output linkages. This framework allows us to trace how tariff shocks propagate throughout the economy and affect different sectors heterogeneously.

\paragraph{} The research addresses several critical questions regarding the welfare consequences of tariff policies. First, we examine the long-term economic impacts of these tariffs on both the U.S. and global economy, considering both direct effects through changed trade costs and indirect effects through sectoral reallocation. Second, we investigate under what conditions tariff policies could potentially benefit the U.S., exploring scenarios where terms-of-trade improvements might offset efficiency losses. Third, we analyze the consequences of retaliatory responses from trading partners, examining how escalating trade tensions affect global welfare and trade patterns.

\paragraph{} Our empirical approach employs several methodological innovations to address these questions comprehensively. We implement a robust parameter estimation strategy using method of moments that jointly estimates iceberg trade costs, wages, and prices while ensuring adherence to trade balance conditions. The framework distinguishes between mobile and immobile labor scenarios, which we interpret as short-run and long-run equilibria respectively, allowing us to examine both immediate and persistent effects of tariff policies. We develop a decomposed structure for iceberg trade costs that reduces computational complexity while maintaining economic interpretability.

\paragraph{} The counterfactual analysis utilizes real-world tariff data from multiple time periods to construct scenarios representing different temporal aggregations and policy designs. This approach enables us to evaluate not only the average effects of tariff changes but also their temporal variation and the optimal design of tariff policies. The analysis covers nine major countries/regions and nine broad economic sectors, providing a comprehensive view of how tariff policies propagate through the interconnected global economy.

\paragraph{} By combining rigorous quantitative methods with detailed empirical data, this research project contributes to our understanding of modern trade policy effectiveness and provides evidence-based insights for policymakers grappling with the complex tradeoffs inherent in tariff policy design.
