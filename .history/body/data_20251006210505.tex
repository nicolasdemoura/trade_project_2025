\section{Data Construction}

This section presents the comprehensive data construction process underlying our multi-sector Ricardian analysis. We integrate multiple international databases to build a consistent analytical framework covering 10 countries and 12 sectors, with particular focus on constructing time-varying tariff scenarios for contemporary US trade policy evaluation during 2024-2025. Our data architecture combines trade data from 2009 with current policy instruments to enable robust counterfactual analysis. Tables \ref{tab:sector_mapping_1} and \ref{tab:sector_mapping_2} present our sector aggregation scheme linking WIOD classifications to HS codes, while Table \ref{tab:country_aggregation} details the country grouping structure.

\subsection{World Input-Output Database (WIOD)}

Our analysis builds on the World Input-Output Database (WIOD) 2013 Release \citep{timmer2015illustrated}, providing comprehensive input-output tables for 27 EU countries and 13 other major economies from 1995 to 2011. We focus on the year 2009 because it offers the most comprehensive trade flow information across our country sample, which is essential for robust estimation of the structural parameters. The WIOD provides three critical data components that form the foundation of our calibration: bilateral trade flows $X_{nik}$ used to compute trade shares $\pi_{nik}$, sectoral intermediate input coefficients $\gamma_{ikk'}$, and final expenditure patterns used to derive consumption shares $\alpha_{nk}$.

We implement a structured aggregation scheme to balance tractability with economic realism. Our 10-country framework includes 8 focus economies (USA, Brazil, China, Japan, Mexico, India, Canada, United Kingdom), the European Union as an integrated bloc (27 member countries), and Rest of World capturing remaining economies. This aggregation captures the primary trade relationships while maintaining computational feasibility, as detailed in Table \ref{tab:country_aggregation}. 

The 56 original WIOD sectors are aggregated into 12 economically meaningful categories following the mapping scheme presented in Tables \ref{tab:sector_mapping_1} and \ref{tab:sector_mapping_2}: Food, Textiles, Paper, Chemical, Metal, Manufacture, Mining, Energy, Construction, Retail/Wholesale, Transport, and Services. This aggregation scheme is designed to capture key sectoral distinctions while maintaining sufficient observations within each category to ensure robust parameter estimation.

\begin{table}[H]
\centering
\caption{Country Aggregation Scheme}
\label{tab:country_aggregation}
\renewcommand{\arraystretch}{1.3}
\resizebox{0.9\textwidth}{!}{%
\begin{tabular}{>{\raggedright}p{3cm} >{\raggedright}p{8cm} >{\raggedright\arraybackslash}p{4cm}}
\toprule
\textbf{Model Region} & \textbf{Constituent Countries/Economies} & \textbf{WIOD Codes} \\
\midrule
\multirow{8}{3cm}{\textbf{Focus Economies}} & United States & USA \\
& Brazil & BRA \\
& China & CHN \\
& Japan & JPN \\
& Mexico & MEX \\
& India & IND \\
& Canada & CAN \\
& United Kingdom & GBR \\
\midrule
\multirow{6}{3cm}{\textbf{European Union}} & \multirow{6}{8cm}{Austria, Belgium, Bulgaria, Croatia, Cyprus, Czech Republic, Denmark, Estonia, Finland, France, Germany, Greece, Hungary, Ireland, Italy, Latvia, Lithuania, Luxembourg, Malta, Netherlands, Poland, Portugal, Romania, Slovakia, Slovenia, Spain, Sweden} & \multirow{6}{4cm}{AUT, BEL, BGR, HRV, CYP, CZE, DNK, EST, FIN, FRA, DEU, GRC, HUN, IRL, ITA, LVA, LTU, LUX, MLT, NLD, POL, PRT, ROU, SVK, SVN, ESP, SWE} \\
& & \\
& & \\
& & \\
& & \\
& & \\
\midrule
\multirow{3}{3cm}{\textbf{Rest of World}} & \multirow{3}{8cm}{Australia, Indonesia, Korea, Russia, Turkey, Taiwan, Other economies} & \multirow{3}{4cm}{AUS, IDN, KOR, RUS, TUR, TWN, RoW} \\
& & \\
& & \\
\bottomrule
\end{tabular}%
}
\begin{tablenotes}
\footnotesize
\item Notes: The aggregation scheme balances analytical tractability with economic realism. Focus economies represent the 8 largest trading partners and policy-relevant countries for our analysis. The European Union is treated as an integrated economic area reflecting deep trade and regulatory integration among member states. Rest of World captures remaining economies in the WIOD database. Country codes follow ISO 3166-1 alpha-3 standard as implemented in \cite{timmer2015illustrated}.
\end{tablenotes}
\end{table}

\begin{table}[H]
\centering
\caption{Sector Mapping: Model Sectors to WIOD and HS Classifications (Part 1)}
\label{tab:sector_mapping_1}
\footnotesize
\renewcommand{\arraystretch}{1.2}
\resizebox{\textwidth}{!}{%
\begin{tabular}{>{\raggedright}p{2.5cm} >{\raggedright}p{4cm} >{\raggedright\arraybackslash}p{10cm}}
\toprule
\multirow{1}{2.5cm}{\textbf{Model Sector}} & \multirow{1}{4cm}{\textbf{WIOD Sectors}} & \multirow{1}{10cm}{\textbf{HS Codes (Chapters)}} \\
\midrule
\multirow{5}{2.5cm}{\textbf{Chemical}} & \multirow{5}{4cm}{Chemicals and Chemical Products (24, c9); Rubber and Plastics (25, c10)} & \multirow{5}{10cm}{28: Inorganic chemicals; 29: Organic chemicals; 30: Pharmaceutical products; 31: Fertilizers; 32: Tanning/dyeing extracts; 33: Essential oils/perfumery; 34: Soap/surface-active agents; 35: Albuminoidal substances; 36: Explosives/pyrotechnics; 38: Miscellaneous chemicals; 39: Plastics} \\
& & \\
& & \\
& & \\
& & \\
\midrule
\multirow{3}{2.5cm}{\textbf{Construction}} & \multirow{3}{4cm}{Construction (F, c18); Real Estate Activities (70, c29)} & \multirow{3}{10cm}{25: Salt/sulphur/earths and stone; 68: Articles of stone/plaster/cement} \\
& & \\
& & \\
\midrule
\multirow{4}{2.5cm}{\textbf{Energy}} & \multirow{4}{4cm}{Coke, Refined Petroleum and Nuclear Fuel (23, c8); Electricity, Gas and Water Supply (E, c17)} & \multirow{4}{10cm}{27: Mineral fuels/oils; 84: Nuclear reactors/boilers/machinery} \\
& & \\
& & \\
& & \\
\midrule
\multirow{6}{2.5cm}{\textbf{Food}} & \multirow{6}{4cm}{Agriculture, Hunting, Forestry and Fishing (AtB, c1); Food, Beverages and Tobacco (15t16, c3)} & \multirow{6}{10cm}{1: Live animals; 2: Meat; 3: Fish/crustaceans; 4: Dairy produce; 5: Animal products; 6: Live trees/plants; 7: Vegetables; 8: Fruit/nuts; 9: Coffee/tea/spices; 10: Cereals; 11: Milling products; 12: Oil seeds; 15: Fats/oils; 16: Meat preparations; 17: Sugars; 18: Cocoa; 19: Cereal preparations; 20: Vegetable preparations; 21: Miscellaneous edible; 22: Beverages; 23: Food waste/animal feed; 24: Tobacco} \\
& & \\
& & \\
& & \\
& & \\
& & \\
\midrule
\multirow{9}{2.5cm}{\textbf{Manufacture}} & \multirow{9}{4cm}{Electrical and Optical Equipment (30t33, c14); Machinery, Nec (29, c13); Manufacturing, Nec; Recycling (36t37, c16); Transport Equipment (34t35, c15)} & \multirow{9}{10cm}{37: Photographic goods; 40: Rubber articles; 41: Raw hides/skins; 42: Leather articles; 43: Furskins; 45: Cork articles; 46: Straw manufactures; 64: Footwear; 65: Headgear; 66: Umbrellas; 67: Feathers; 69: Ceramics; 70: Glass; 71: Precious stones; 82: Tools/cutlery; 83: Miscellaneous base metal; 85: Electrical machinery; 86: Railway vehicles; 87: Motor vehicles; 88: Aircraft; 89: Ships; 90: Optical instruments; 91: Clocks/watches; 92: Musical instruments; 93: Arms/ammunition; 94: Furniture; 95: Toys/games; 96: Miscellaneous manufactures; 97: Art/antiques} \\
& & \\
& & \\
& & \\
& & \\
& & \\
& & \\
& & \\
& & \\
\bottomrule
\end{tabular}%
}
\begin{tablenotes}
\footnotesize
\item Notes: WIOD sector codes in parentheses show both the alphanumeric (first) and numeric (second) identifiers from \cite{timmer2015illustrated}. HS codes refer to 2-digit Harmonized System chapters. Table continues on next page.
\end{tablenotes}
\end{table}

\begin{table}[H]
\centering
\caption{Sector Mapping: Model Sectors to WIOD and HS Classifications (Part 2)}
\label{tab:sector_mapping_2}
\footnotesize
\renewcommand{\arraystretch}{1.2}
\resizebox{\textwidth}{!}{%
\begin{tabular}{>{\raggedright}p{2.5cm} >{\raggedright}p{4cm} >{\raggedright\arraybackslash}p{10cm}}
\toprule
\multirow{1}{2.5cm}{\textbf{Model Sector}} & \multirow{1}{4cm}{\textbf{WIOD Sectors}} & \multirow{1}{10cm}{\textbf{HS Codes (Chapters)}} \\
\midrule
\multirow{4}{2.5cm}{\textbf{Metal}} & \multirow{4}{4cm}{Basic Metals and Fabricated Metal (27t28, c12); Other Non-Metallic Mineral (26, c11)} & \multirow{4}{10cm}{72: Iron/steel; 73: Iron/steel articles; 74: Copper; 75: Nickel; 76: Aluminium; 78: Lead; 80: Tin; 81: Other base metals; 79: Zinc} \\
& & \\
& & \\
& & \\
\midrule
\multirow{2}{2.5cm}{\textbf{Mining}} & \multirow{2}{4cm}{Mining and Quarrying (C, c2)} & \multirow{2}{10cm}{26: Ores/slag/ash; 13: Lac/gums/resins; 14: Vegetable plaiting materials} \\
& & \\
\midrule
\multirow{4}{2.5cm}{\textbf{Paper}} & \multirow{4}{4cm}{Pulp, Paper, Printing and Publishing (21t22, c7); Wood and Products of Wood and Cork (20, c6)} & \multirow{4}{10cm}{44: Wood/wood articles; 47: Wood pulp; 48: Paper/paperboard} \\
& & \\
& & \\
& & \\
\midrule
\multirow{5}{2.5cm}{\textbf{Retail and Wholesale}} & \multirow{5}{4cm}{Retail Trade, Except Motor Vehicles (52, c21); Sale/Maintenance of Motor Vehicles (50, c19); Wholesale Trade (51, c20)} & \multirow{5}{10cm}{\textit{Non-tradable services sector}} \\
& & \\
& & \\
& & \\
& & \\
\midrule
\multirow{13}{2.5cm}{\textbf{Services}} & \multirow{13}{4cm}{Education (M, c32); Financial Intermediation (J, c28); Health and Social Work (N, c33); Hotels and Restaurants (H, c22); Other Community/Social/Personal Services (O, c34); Post and Telecommunications (64, c27); Private Households with Employed Persons (P, c35); Public Admin and Defence (L, c31); Renting of M\&Eq and Other Business Activities (71t74, c30)} & \multirow{13}{10cm}{\textit{Non-tradable services sector}} \\
& & \\
& & \\
& & \\
& & \\
& & \\
& & \\
& & \\
& & \\
& & \\
& & \\
& & \\
& & \\
\midrule
\multirow{4}{2.5cm}{\textbf{Textiles}} & \multirow{4}{4cm}{Leather, Leather and Footwear (19, c5); Textiles and Textile Products (17t18, c4)} & \multirow{4}{10cm}{50: Silk; 51: Wool/animal hair; 52: Cotton; 53: Other vegetable fibers; 54: Man-made filaments; 55: Man-made staple fibers; 56: Wadding/felt; 57: Carpets; 58: Special woven fabrics; 59: Impregnated textiles; 60: Knitted fabrics; 61: Knitted apparel; 62: Woven apparel; 63: Other textiles} \\
& & \\
& & \\
& & \\
\midrule
\multirow{5}{2.5cm}{\textbf{Transport}} & \multirow{5}{4cm}{Air Transport (62, c25); Inland Transport (60, c23); Other Supporting Transport Activities (63, c26); Water Transport (61, c24)} & \multirow{5}{10cm}{\textit{Non-tradable services sector}} \\
& & \\
& & \\
& & \\
& & \\
\bottomrule
\end{tabular}%
}
\begin{tablenotes}
\footnotesize
\item Notes: WIOD sector codes in parentheses show both the alphanumeric (first) and numeric (second) identifiers from \cite{timmer2015illustrated}. HS codes refer to 2-digit Harmonized System chapters. Non-tradable services sectors do not have corresponding HS classifications as they represent domestic activities.
\end{tablenotes}
\end{table}


\subsection{Socioeconomic Accounts and Labor Data}

Labor market data comes from the WIOD Socioeconomic Accounts (SEA) July 2014 release \citep{timmer2015illustrated}, providing employment and compensation data by country and sector for 2009. 

All monetary values are converted to 2009 US dollars using annual average exchange rates from the IMF International Financial Statistics, ensuring cross-country comparability. The WIOD input-output framework enables direct computation of key structural parameters: labor shares $\beta_{ik} = \frac{\text{Labor Compensation}_{ik}}{\text{Gross Output}_{ik}}$ capture the factor intensity of production consistent with our Cobb-Douglas specification. Together with intermediate input coefficients $\gamma_{ikk'} = \frac{\text{Intermediate Purchases}_{ikk'}}{\text{Gross Output}_{ik}}$ and final expenditure shares $\alpha_{nk} = \frac{\text{Final Consumption}_{nk}}{\text{Total Final Consumption}_{n}}$, these parameters form the technological and preference foundations of our structural model, estimated from observed 2009 production and consumption patterns.

\subsection{Tariff Data Architecture}

Our empirical analysis employs a dual-source approach to capture both comprehensive international coverage and high-frequency US policy dynamics. We integrate tariff data from the TRAINS database for international comparability with detailed USITC customs data for contemporary US trade policy analysis, enabling robust counterfactual evaluation of recent policy changes while maintaining structural consistency across our multi-country framework.

The TRAINS database, accessed through the World Bank's WITS platform \citep{WITS2025}, provides our baseline tariff structure for all non-US economies in the sample. We utilize 2023 data as the most recent year with comprehensive coverage across our 9 international partners (Brazil, China, Japan, Mexico, India, Canada, United Kingdom, European Union, and Rest of World). The TRAINS system offers ad valorem equivalent rates at the HS 6-digit level, which we aggregate to match our 12-sector classification scheme. This database ensures consistent measurement methodology across countries and provides the structural foundation for calibrating bilateral trade costs in our baseline equilibrium.

For the United States, we employ detailed HTS-level data from USITC DataWeb covering 2024-2025 \citep{USITC2025}. This high-frequency dataset contains monthly observations of General Customs Value and General Import Charges at the 10-digit HTS level, enabling precise calculation of effective tariff rates as $\tau_{ikt} = \frac{\text{Import Charges}_{ikt}}{\text{Customs Value}_{ikt}}$ for each trading partner and product category. The USITC data captures recent policy implementations, including antidumping duties, countervailing duties, and Section 301 tariffs, providing the granular information necessary for evaluating contemporary trade policy scenarios. In order to simplify, we consider this implied tariff as the total tariff faced by US importers, ignoring potential interactions with other trade barriers.

We calibrate the structural model using 2023 tariff data as our baseline equilibrium for international partners, combined with 2024 US rates as the initial state for counterfactual analysis. For policy evaluation, we construct three temporal aggregations of 2025 US tariff policies: a rolling 12-month window to capture medium-term policy trends, year-to-date averages through the most recent available month, and the latest quarterly period to identify short-term policy shifts. While the quarterly (3-month) aggregation provides the most accurate representation of current policy stance, it suffers from seasonal fluctuations in trade patterns that can distort effective tariff calculations. Extending to a 12-month rolling window mitigates these seasonal effects but incorporates pre-2025 tariff policies that may not reflect the current administration's trade stance. This approach maintains international comparability through TRAINS while leveraging USITC's superior temporal resolution for US policy analysis, with the quarterly measure serving as our primary specification despite its seasonal limitations.
