\section{Introduction}

This project quantifies the welfare effects of contemporary U.S. tariff policies using structural estimation of a multi-sector general equilibrium trade model. While standard trade theory predicts that unilateral tariffs reduce global welfare, the magnitude of these losses and their distribution across countries and sectors remains an empirical question requiring careful quantitative analysis.

We employ the \cite{costinot2012TheReviewofEconomicStudies} framework to evaluate tariff impacts through structural estimation. The CDK model's multi-sector structure captures sectoral interdependencies via input-output linkages, enabling analysis of how tariff shocks propagate across industries within and between countries. This approach provides precise welfare measurements by accounting for general equilibrium adjustments in prices, wages, and trade flows.

Our analysis addresses three key questions. First, what are the aggregate welfare effects of recent U.S. tariff increases on domestic and international economies? Second, how do these effects vary across sectors and countries depending on trade linkages and comparative advantage patterns? Third, how do short-run adjustments (immobile labor) compare with long-run equilibrium outcomes (mobile labor)?

The empirical strategy combines WIOD input-output data from 2009 with contemporary HTS-level U.S. tariff data from 2024-2025. We estimate the model using method of moments, targeting observed bilateral trade shares while satisfying trade balance and price consistency constraints. Trade costs are parameterized using a decomposed structure that separates bilateral, importer-specific, and exporter-specific components, reducing computational complexity while maintaining economic interpretability.

We construct three temporal tariff scenarios—12-month rolling averages, year-to-date rates, and quarterly recent data—to evaluate policy persistence versus temporary adjustments. The analysis covers 10 countries/regions and 12 sectors, providing comprehensive coverage of major trading relationships and economic activities.

Results show that contemporary U.S. tariff policies generate welfare losses ranging from 0.34-0.41% under mobile labor and 0.52-0.59% under immobile labor, with significant heterogeneity across countries and sectors. These findings contribute quantitative evidence on trade policy effectiveness and inform policy design in an era of renewed protectionism.
