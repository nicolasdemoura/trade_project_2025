\section{Conclusion}

This project quantifies the welfare effects of contemporary U.S. tariff policies using structural estimation of the \cite{costinot2012TheReviewofEconomicStudies} multi-sector trade model. By integrating 2009 WIOD data with 2024-2025 HTS-level tariff information, we provide quantitative evidence on the costs and distributional consequences of recent protectionist measures.

\subsection{Main Results}

The welfare analysis reveals asymmetric impacts across countries and time horizons. Canada and Mexico emerge as primary losers from U.S. tariff increases, experiencing welfare losses that exceed their direct tariff exposure due to supply chain disruptions and reduced competitiveness within the integrated North American economy. The United States experiences modest welfare gains through terms-of-trade improvements, while the EU benefits substantially from trade diversion effects.

Temporal variation in tariff scenarios indicates that recent quarterly measures impose higher welfare costs than longer-term averages, suggesting that current policies may represent temporary departures rather than permanent shifts. The concentration of welfare effects in five key sectors (Chemicals, Textiles, Metals, Manufacturing, Energy) highlights the targeted nature of contemporary trade policy.

Parameter estimates confirm expected patterns: developing countries specialize in labor-intensive sectors and allocate higher budget shares to food, while developed economies focus on skill-intensive services and maintain higher construction expenditure. Productivity estimates align with comparative advantage theory, and trade costs display intuitive geographic patterns with preferential arrangements reducing bilateral barriers.

\subsection{Policy Implications}

Results indicate that unilateral tariff increases generate net costs for the implementing country while creating winners and losers internationally through trade diversion. The magnitude of welfare effects—ranging from 0.34% to 0.41% of national income—suggests substantial economic consequences that should be weighed against any non-economic policy objectives.

The distributional heterogeneity across sectors and countries emphasizes the importance of considering adjustment costs in policy design. Protected sectors benefit while export-oriented industries face reductions, creating internal redistributive tensions that may require complementary policies.

\subsection{Limitations}

Several limitations constrain our analysis. The immobile labor scenario produces inconsistent results that suggest specification issues, limiting confidence in short-run welfare estimates. The model's poor fit to GDP patterns indicates structural limitations in capturing all general equilibrium adjustments. Future research could explore alternative parameterizations or incorporate dynamic adjustment mechanisms to better represent transition costs and factor reallocation processes.

Several limitations suggest directions for future research. First, our three-country aggregation, while computationally tractable, may miss important bilateral relationships with specific major trading partners such as China and Mexico. Future work could extend the analysis to include more disaggregated country coverage.

Second, our focus on tariff policy abstracts from other trade policy instruments such as non-tariff barriers, trade facilitation measures, and regulatory harmonization. A comprehensive trade policy evaluation would incorporate these additional dimensions.

Third, our static framework does not capture dynamic effects such as investment responses, innovation incentives, or learning-by-doing effects that might modify the welfare calculations over longer time horizons. Dynamic extensions would provide valuable insights into the long-run effects of trade policy changes.

Fourth, our welfare analysis focuses on aggregate real income effects and does not incorporate non-economic objectives that might justify protectionist policies, such as national security, supply chain resilience, or strategic industrial development. Future research might develop frameworks for incorporating these objectives into quantitative policy evaluation.

\subsection{Final Remarks}

Contemporary debates over trade policy would benefit from rigorous quantitative analysis using state-of-the-art theoretical and empirical methods. Our framework provides a foundation for evidence-based policy evaluation that incorporates both theoretical rigor and empirical realism. As trade policy continues to evolve in response to geopolitical and economic challenges, maintaining the capacity for timely and accurate policy evaluation becomes increasingly important for informed democratic deliberation.

The integration of high-frequency administrative data with structural economic models represents a promising direction for policy-relevant research. Our approach demonstrates that it is possible to provide quantitative policy guidance that is both theoretically grounded and empirically current, bridging the gap between academic research and policy application that has often limited the practical influence of economic analysis.

The welfare costs we identify are substantial enough to warrant careful consideration in policy deliberations. While our analysis does not incorporate all potential benefits of trade policy changes, it provides a baseline for evaluating whether such benefits are sufficient to justify the measured economic costs. This quantitative foundation can inform more effective and efficient trade policy design in an increasingly complex global economy.
