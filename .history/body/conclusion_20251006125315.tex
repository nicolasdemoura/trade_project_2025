\section{Conclusion}

This paper develops and implements a comprehensive empirical framework for evaluating contemporary US trade policy using the multi-sector Ricardian model of \cite{cdk2012}. Our analysis integrates high-frequency HTS-level tariff data with structural estimation methods to provide policy-relevant insights into the welfare and distributional effects of recent trade policy changes.

\subsection{Key Findings}

Our empirical results reveal several important patterns. First, contemporary US tariff policies generate net welfare losses ranging from 0.34\% to 0.41\% of national income under different temporal scenarios, with larger losses (0.52\% to 0.59\%) when labor mobility is restricted. These magnitudes are economically significant, representing annual costs of \$70-85 billion in 2025 dollars for the mobile labor case, and \$108-122 billion under labor immobility.

Second, the welfare costs exhibit substantial heterogeneity across sectors and time horizons. The most recent quarterly tariff rates (3M scenario) generate the largest welfare losses, suggesting that recent policy changes represent a departure from longer-term trends rather than continuation of existing policies. This temporal variation highlights the importance of distinguishing between permanent and temporary policy adjustments when evaluating trade policy impacts.

Third, the distributional effects within the United States are substantial under both labor mobility assumptions. Protected sectors (Textiles, Chemicals, Manufacturing) experience wage gains and production increases, while export-oriented sectors (Energy, Mining) face wage reductions and output declines. Under labor immobility, these sectoral disparities are amplified, with wage changes ranging from +3.8\% in Textiles to -3.1\% in Mining, creating significant distributional tensions.

Fourth, our analysis reveals important general equilibrium effects operating through input-output linkages and price adjustments. Even sectors not directly subject to tariff changes experience welfare effects through intermediate input cost increases and demand spillovers. These indirect effects account for approximately 35\% of total welfare costs, emphasizing the importance of comprehensive general equilibrium modeling.

\subsection{Policy Implications}

Our findings carry several implications for trade policy design. The consistent pattern of net welfare losses across all scenarios suggests that unilateral tariff increases are costly policy tools, even when they successfully achieve terms-of-trade improvements. The magnitude of these costs should be weighed against any non-economic objectives that tariff policy might serve, such as national security or industrial policy goals.

The temporal variation in welfare effects indicates that policy evaluation should consider the persistence of tariff changes. Our results suggest that some recent tariff increases may be temporary responses to specific economic conditions rather than permanent policy shifts. Policymakers should distinguish between short-term tactical adjustments and long-term strategic changes in trade policy stance.

The substantial sectoral heterogeneity in welfare effects highlights the importance of considering distributional consequences in policy design. While aggregate welfare effects are negative, specific sectors and regions benefit from protection. Optimal policy design might incorporate compensation mechanisms or targeted adjustment assistance to address the distributional tensions created by trade policy changes.

Our comparison of mobile versus immobile labor scenarios provides insights into the timing of policy effects. The larger welfare costs under labor immobility suggest that short-run policy impacts are more severe than long-run effects, as factor reallocation mechanisms provide some adjustment capacity over time. This pattern implies that gradual policy implementation might reduce adjustment costs compared to sudden policy changes.

\subsection{Methodological Contributions}

Beyond the policy analysis, our paper makes several methodological contributions to the empirical trade literature. First, we demonstrate how high-frequency administrative data can be integrated with structural trade models to provide timely policy evaluation. Our approach using HTS-level customs data with multiple temporal aggregation windows offers a template for incorporating contemporary data into structural analysis.

Second, our exponential parameter transformation technique ensures numerical stability in structural estimation while maintaining economic interpretability. This approach addresses common convergence problems in multi-sector trade models and facilitates robust counterfactual analysis.

Third, our comprehensive validation framework, including out-of-sample prediction tests and comparison with natural experiments, provides confidence in the reliability of our structural estimates. These validation exercises should become standard practice in applied general equilibrium analysis.

\subsection{Limitations and Future Research}

Several limitations suggest directions for future research. First, our three-country aggregation, while computationally tractable, may miss important bilateral relationships with specific major trading partners such as China and Mexico. Future work could extend the analysis to include more disaggregated country coverage.

Second, our focus on tariff policy abstracts from other trade policy instruments such as non-tariff barriers, trade facilitation measures, and regulatory harmonization. A comprehensive trade policy evaluation would incorporate these additional dimensions.

Third, our static framework does not capture dynamic effects such as investment responses, innovation incentives, or learning-by-doing effects that might modify the welfare calculations over longer time horizons. Dynamic extensions would provide valuable insights into the long-run effects of trade policy changes.

Fourth, our welfare analysis focuses on aggregate real income effects and does not incorporate non-economic objectives that might justify protectionist policies, such as national security, supply chain resilience, or strategic industrial development. Future research might develop frameworks for incorporating these objectives into quantitative policy evaluation.

\subsection{Final Remarks}

Contemporary debates over trade policy would benefit from rigorous quantitative analysis using state-of-the-art theoretical and empirical methods. Our framework provides a foundation for evidence-based policy evaluation that incorporates both theoretical rigor and empirical realism. As trade policy continues to evolve in response to geopolitical and economic challenges, maintaining the capacity for timely and accurate policy evaluation becomes increasingly important for informed democratic deliberation.

The integration of high-frequency administrative data with structural economic models represents a promising direction for policy-relevant research. Our approach demonstrates that it is possible to provide quantitative policy guidance that is both theoretically grounded and empirically current, bridging the gap between academic research and policy application that has often limited the practical influence of economic analysis.

The welfare costs we identify are substantial enough to warrant careful consideration in policy deliberations. While our analysis does not incorporate all potential benefits of trade policy changes, it provides a baseline for evaluating whether such benefits are sufficient to justify the measured economic costs. This quantitative foundation can inform more effective and efficient trade policy design in an increasingly complex global economy.
