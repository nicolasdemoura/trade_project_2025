Solve for the closed economy values of $\varphi^*$, $P$, $R$.

\begin{solution}
    \paragraph{Household side} The household's problem is given by:
    \begin{align*}
        \max_{q(\omega)} \quad & U = \left( \int_{\omega \in \Omega} q(\omega)^{\rho} d\omega \right)^{\frac{1}{\rho}} \\
        \text{s.t.} \quad & \int_{\omega \in \Omega} p(\omega) q(\omega) d\omega = R
    \end{align*}
    where $\rho = \frac{\sigma - 1}{\sigma}$ and $\sigma > 1$. The Lagrangian for this problem is given by:
    \begin{align*}
        \mathcal{L} = \left( \int_{\omega \in \Omega} q(\omega)^{\rho} d\omega \right)^{\frac{1}{\rho}} + \lambda \left( R - \int_{\omega \in \Omega} p(\omega) q(\omega) d\omega \right)
    \end{align*}
    The first order condition with respect to $q(\omega)$ is given by:
    \begin{align*}
        \frac{\partial \mathcal{L}}{\partial q(\omega)} = 0 &\iff \frac{1}{\rho} \left( \int_{\omega \in \Omega} q(\omega)^{\rho} d\omega \right)^{\frac{1}{\rho} - 1} \rho q(\omega)^{\rho - 1} - \lambda p(\omega) = 0 \\
        &\iff q(\omega)^{\rho - 1} = \lambda p(\omega) \left( \int_{\omega \in \Omega} q(\omega)^{\rho} d\omega \right)^{1 - \frac{1}{\rho}} \\
        &\iff q(\omega) = \left[ \lambda p(\omega) \right]^{\frac{1}{\rho - 1}} \left( \int_{\omega \in \Omega} q(\omega)^{\rho} d\omega \right)^{\frac{1}{\rho} } \\
        &\iff q(\omega) = \left[ \lambda p(\omega) \right]^{-\sigma} Q 
    \end{align*}
    where $Q = \left( \int_{\omega \in \Omega} q(\omega)^{\rho} d\omega \right)^{\frac{1}{\rho}}$ is the CES aggregate. Substituting this expression for $q(\omega)$ into the budget constraint, we have:
    \begin{align*}
        R &= \int_{\omega \in \Omega} p(\omega) q(\omega) d\omega \\
        &= \int_{\omega \in \Omega} p(\omega) \left[ \lambda p(\omega) \right]^{-\sigma} Q d\omega \\
        &= Q \lambda^{-\sigma} \int_{\omega \in \Omega} p(\omega)^{1 - \sigma} d\omega \\
        &\iff \lambda^{-\sigma} = \frac{R}{Q P^{1 - \sigma}} 
    \end{align*}
    where $P = \left( \int_{\omega \in \Omega} p(\omega)^{1 - \sigma} d\omega \right)^{\frac{1}{1 - \sigma}}$ is the CES price index. Substituting this expression for $\lambda^{-\sigma}$ back into the expression for $q(\omega)$, we have:
    \begin{align*}
        q(\omega) &= \left[ \frac{R}{Q P^{1 - \sigma}} p(\omega) \right]^{-\sigma} Q \\
        &\iff q(\omega) = \frac{R}{P} \left( \frac{p(\omega)}{P} \right)^{-\sigma}\\
        &\iff q(\omega) = Q \left( \frac{p(\omega)}{P} \right)^{-\sigma}
    \end{align*}
    The revenue associated with a variety $\omega$ is given by:
    \begin{align*}
        r(\omega) &= p(\omega) q(\omega) \\
        &= p(\omega) \frac{R}{P} \left( \frac{p(\omega)}{P} \right)^{-\sigma} \\
        &\iff r(\omega) = R \left( \frac{p(\omega)}{P} \right)^{1 - \sigma}
    \end{align*}
    
    \paragraph{Producer's Problem.} A firm with productivity $\varphi$ needs $\dfrac{1}{\varphi}$ units of labor to produce one unit of output. It also pays a fixed cost $f$ units of labor to operate. Hence, the total cost of producing quantity $q$ is given by:
    \begin{align*}
        C(q, \varphi) = f + \frac{q}{\varphi} 
    \end{align*}
    
    The profit maximization problem for the firm is given by:
    \begin{align*}
        \max_{p(\varphi)} \quad & \pi(\varphi) = r(\varphi) - w\left( f + \frac{q(\varphi)}{\varphi} \right)
    \end{align*}
    where $w$ is the wage rate. Normalizing the wage to $w = 1$, the profit maximization problem becomes:
    \begin{align*}
        \max_{p(\varphi)} \quad & \pi(\varphi) = r(\varphi) - \left( f + \frac{q(\varphi)}{\varphi} \right)
    \end{align*}
    The FOC with respect to $p(\varphi)$ is given by:
    \begin{align*}
        \frac{\partial \pi(\varphi)}{\partial p(\varphi)} = 0 &\iff \frac{\partial r(\varphi)}{\partial p(\varphi)} - \frac{1}{\varphi} \frac{\partial q(\varphi)}{\partial p(\varphi)} = 0 \\
        &\iff R(1 - \sigma) \left( \frac{p(\varphi)}{P} \right)^{-\sigma} \frac{1}{P} - \frac{1}{\varphi} \left( -\sigma Q \left( \frac{p(\varphi)}{P} \right)^{-\sigma - 1} \frac{1}{P} \right) = 0 \\
        &\iff (1 - \sigma) r(\varphi) + \frac{\sigma}{\varphi} q(\varphi) = 0 \\
        &\iff r(\varphi) = \frac{\sigma}{\sigma - 1} \frac{1}{\varphi} q(\varphi) \\
        &\iff p(\varphi) = \frac{\sigma}{\sigma - 1} \frac{1}{\varphi} = \frac{1}{\rho\varphi} 
    \end{align*}
    where we used the demand result above to substitute for $q(\varphi)$ in the third line. Thus, the optimal price charged by a firm with productivity $\varphi$ is a constant markup over marginal cost, as expected.
    
    Substituting this into the revenue expression and using the demand result above,
    \begin{align*}
        r(\varphi) &= R \left( \frac{p(\varphi)}{P} \right)^{1 - \sigma} \\
        &= R \left( \frac{1}{\rho \varphi P} \right)^{1 - \sigma} \\
        &= (\rho \varphi)^{\sigma - 1} R P^{\sigma - 1}
    \end{align*} 
    Profits from serving the domestic market are: 
    \begin{align*}
        \pi(\varphi) &= r(\varphi) - \left( f + \frac{q(\varphi)}{\varphi} \right) \\
        &= r(\varphi) - f - \frac{1}{\varphi} Q \left( \frac{p(\varphi)}{P} \right)^{-\sigma} \\
        &= r(\varphi) - f - \frac{1}{\varphi} Q \left( \frac{1}{\rho\varphi P} \right)^{-\sigma} \\
        &= r(\varphi) - f -  \rho^{\sigma} \varphi^{\sigma - 1} R P^{\sigma - 1} \\
        &= r(\varphi) - f -  \rho r(\varphi) \\
        &= \frac{1}{\sigma} r(\varphi) - f
    \end{align*}
    
    The marginal (least productive) active firm satisfies $\pi(\varphi^*)=0$, hence:
    \begin{align*}
        \pi(\varphi^*) = 0 &\iff \frac{1}{\sigma} r(\varphi^*) - f = 0 \\
        &\iff r(\varphi^*) = \sigma f 
    \end{align*}

    \paragraph{Enter and Exit.} Firms pay a sunk entry cost $f_e$ to draw a productivity $\varphi$ from the distribution $G$. If they choose to operate, they pay the fixed cost $f$ and produce optimally as above. If they do not operate, they exit and earn zero profits. The probability of drawing a productivity above the cutoff $\varphi^*$ is $1 - G(\varphi^*)$. Hence, the ex-post productivity distribution among active firms is given by the truncated distribution:
    \begin{align*}
        \mu(\varphi) = \begin{cases}
            \frac{g(\varphi)}{1 - G(\varphi^*)} & \text{if } \varphi \geq \varphi^*\\
            0 & \text{otherwise}
        \end{cases}
    \end{align*}
    where $g(\varphi)$ is the density associated with $G$.

    After entering the market, there is a probability $\delta$ that the firm exits exogenously. Hence, the value of a firm with productivity $\varphi$ is given by:
    \begin{align*}\
        v(\varphi) = \max\{\frac{\pi(\varphi)}{\delta}, 0\}
    \end{align*}
    The free entry condition requires that the expected value of entry equals the entry cost:
    \begin{align*}
        \int v(\varphi) d\mu(\varphi) = f_e &\iff \int_{\varphi^*}^{\infty} \frac{\pi(\varphi)}{\delta} \frac{g(\varphi)}{1 - G(\varphi^*)} d\varphi = f_e \\
        &\iff \frac{1-G(\varphi^*)}{\delta} \int_{\varphi^*}^{\infty} \pi(\varphi) g(\varphi) d\varphi = f_e \\
        &\iff v_e = \frac{1-G(\varphi^*)}{\delta} \bar{\pi} = f_e
    \end{align*}
    where $\bar{\pi} = \int_{\varphi^*}^{\infty} \pi(\varphi) g(\varphi) d\varphi$ is the average profit among active firms and $v_e$ is the expected value of entry.
    
    \bigskip
    	\textbf{Aggregate price index (CES aggregator).}
    
    Using $p(\varphi)^{1-\sigma}=\big(\tfrac{\sigma}{\sigma-1}\big)^{1-\sigma} \varphi^{\sigma-1}$, the CES price index is
    \[
        P^{1-\sigma} \,=\, \int_{\varphi\ge \varphi^*} p(\varphi)^{1-\sigma}\, d\mu(\varphi)
        \,=\, \Big(\frac{\sigma}{\sigma-1}\Big)^{1-\sigma}\, M\, \mathbb{E}\big[\varphi^{\sigma-1}\,\big|\,\varphi\ge \varphi^*\big],
    \]
    where $M$ is the mass of active domestic varieties and the expectation is taken over the productivity distribution truncated at $\varphi^*$. Equivalently,
    \[
        \boxed{\;\displaystyle P \,=\, \Big[\Big(\frac{\sigma}{\sigma-1}\Big)^{1-\sigma}\, M\, \mathbb{E}\big(\varphi^{\sigma-1}\,\big|\,\varphi\ge \varphi^*\big)\Big]^{\!\tfrac{1}{1-\sigma}}\; }.
    \]
    
    \bigskip
    	\textbf{Pareto productivity hypothesis and closed forms.}

    Assume productivities are i.i.d. Pareto on $[\varphi_m,\infty)$ with tail parameter $\theta>\sigma-1$:
    \[
        G(\varphi)=1-\Big(\tfrac{\varphi_m}{\varphi}\Big)^{\!\theta},\qquad g(\varphi)=\theta\,\varphi_m^{\theta}\,\varphi^{-\theta-1},\quad \varphi\ge\varphi_m.
    \]
    For any $a\ge\varphi_m$, the truncated $(\sigma-1)$-moment is
    \[
        \mathbb{E}\big[\varphi^{\sigma-1}\,\big|\,\varphi\ge a\big] \,=\, \frac{\theta}{\theta-(\sigma-1)}\, a^{\sigma-1}.
    \]
    Using this inside the price-index formula yields
    \[
        P^{1-\sigma} \,=\, \Big(\tfrac{\sigma}{\sigma-1}\Big)^{1-\sigma} M \frac{\theta}{\theta-(\sigma-1)}\, (\varphi^*)^{\sigma-1}.
    \]
    From the cutoff condition $R\Big(\tfrac{\sigma}{\sigma-1}\Big)^{1-\sigma} (\varphi^* P)^{\sigma-1}=\sigma f$, we have
    \[
        (\varphi^*)^{\sigma-1} \,=\, \frac{\sigma f}{R}\Big(\tfrac{\sigma}{\sigma-1}\Big)^{\!\sigma-1} P^{1-\sigma}.
    \]
    Substituting and simplifying cancels $P^{1-\sigma}$ and markup terms on both sides and delivers the identity (no circular reference):
    \[
        \boxed{\; R \,=\, \frac{\sigma\,\theta}{\theta-(\sigma-1)}\, M f \; }.
    \]
    One can also write the joint level $(\varphi^* P)$ directly from the cutoff condition as
    \[
        \boxed{\; (\varphi^* P)^{\sigma-1} \,=\, \Big(\tfrac{\sigma}{\sigma-1}\Big)^{\!\sigma-1} \frac{\sigma f}{R} \; }.
    \]

    \bigskip
    	\textbf{Total revenue $R$ (labor-income closure, optional).}

    With only labor and $w=1$, free entry drives pure profits to zero so $R=wL=L$. Combining with the Pareto identity above also gives $M=\dfrac{\theta-(\sigma-1)}{\sigma\theta}\,\dfrac{R}{f}$ if desired.
    
    
    \bigskip
    	\textbf{Final closed-economy answers (in your notation):}
    \[
    \boxed{\;\displaystyle \varphi^* \,=\, \frac{\sigma}{\sigma-1}\,\Big(\frac{\sigma f}{R}\Big)^{\!\frac{1}{\sigma-1}} \frac{1}{P}\; },
    \quad
    \boxed{\;\displaystyle P \,=\, \Big[\Big(\frac{\sigma}{\sigma-1}\Big)^{1-\sigma}\, M\, \mathbb{E}\big(\varphi^{\sigma-1}\,\big|\,\varphi\ge \varphi^*\big)\Big]^{\!\tfrac{1}{1-\sigma}}\; },
    \quad
    \boxed{\; R = L\; (w=1) },\quad \boxed{\; R \,=\, \dfrac{\sigma\,\theta}{\theta-(\sigma-1)}\, M f\; }.
    \]
\end{solution}