\section{Data Construction}

This section describes the comprehensive data construction process for our multi-sector Ricardian analysis. We combine multiple data sources to construct a consistent dataset covering 30 countries and 12 sectors, with particular attention to tariff scenarios for 2024-2025 policy analysis. Table \ref{tab:data_summary} provides an overview of our primary data sources.

\subsection{World Input-Output Database (WIOD)}

Our analysis builds on the 2016 release of the World Input-Output Database (WIOD), providing comprehensive input-output tables for 43 countries and 56 sectors from 1995 to 2014. Following the literature standard, we focus on the year 2009 to avoid distortions from the global financial crisis while maintaining sufficient data coverage. The WIOD provides three critical data components: bilateral trade flows $X_{nik}$, intermediate input coefficients $\gamma_{ikk'}$, and final consumption patterns $\alpha_{nk}$.

We implement a structured aggregation scheme to balance tractability with economic realism. Countries are aggregated into three groups: United States (our focal economy), European Union (27 member countries), and Rest of World (7 major economies plus aggregate). This aggregation captures the primary trade relationships while maintaining computational feasibility. The 56 original WIOD sectors are aggregated into 12 economically meaningful categories: Food, Textiles, Paper, Chemical, Metal, Manufacture, Mining, Energy, Construction, Retail/Wholesale, Transport, and Services.

The aggregation preserves key economic relationships by maintaining consistent input-output coefficients and ensuring that trade balance conditions hold at the aggregated level. We verify that $\sum_{i} \sum_{k} X_{nik} = \sum_{i} \sum_{k} \pi_{nik} X_{nk}$ for all countries $n$, where $\pi_{nik}$ represents the trade share of country $n$'s imports from country $i$ in sector $k$.

\subsection{Socioeconomic Accounts and Labor Data}

Labor market data comes from the WIOD Socioeconomic Accounts (SEA), providing employment $L_{nk}$ and wage data $w_{nk}$ by country and sector. We use total hours worked rather than employment counts to capture differences in work intensity across countries and sectors. This choice is particularly important for services sectors where part-time employment varies significantly across countries.

Exchange rate adjustments use annual average rates from the IMF International Financial Statistics to convert all monetary values to 2009 US dollars. This ensures comparability across countries while avoiding short-term volatility that might distort structural parameters. Labor shares $\beta_{ik}$ are computed directly from the input-output tables as the ratio of wage payments to gross output, ensuring consistency with the production function specification in equation (\ref{eq:production}).

\subsection{Tariff Data Architecture}

Our empirical analysis employs a dual-source approach to capture both historical patterns and contemporary policy dynamics. The baseline tariff structure uses the TRAINS database through WITS, providing ad valorem equivalent rates for 2009 that match our WIOD reference year. These historical tariffs establish the structural relationships and serve as the counterfactual baseline.

For policy-relevant analysis, we integrate a comprehensive HTS-level dataset covering US imports for 2024-2025. This dataset contains monthly observations of General Customs Value and General Import Charges at the 10-digit HTS level, allowing us to compute effective tariff rates as $\tau_{nik} = \frac{\text{Import Charges}}{\text{Customs Value}}$ for each trading partner and product category.

The HTS data enables construction of four temporal tariff scenarios: (1) \texttt{tariff\_rate24} uses 2024 annual averages as the baseline, (2) \texttt{tariff\_rate25\_YTD} covers year-to-date 2025 through the most recent month, (3) \texttt{tariff\_rate25\_LTM} uses a rolling 12-month window ending with the latest data, and (4) \texttt{tariff\_rate25\_3M} captures the most recent quarterly trends. This temporal structure allows analysis of evolving trade policies and seasonal patterns.

HTS codes are mapped to our 12-sector classification using concordance tables based on 2-digit HS categories. For example, HS chapters 28-39 map to Chemicals, chapters 84-85 to Energy/Machinery, and chapters 01-24 to Food products. Country mapping aggregates individual trading partners into our three-region framework based on trade volumes and economic integration patterns. Trade-weighted averaging ensures that tariff rates reflect the economic importance of different products and partners within each sector.

\subsection{Data Integration and Validation}

The final dataset integrates all sources into a consistent framework covering countries $n \in \{$USA, EU, RoW$\}$ and sectors $k \in \{1, ..., 12\}$. We implement several validation procedures to ensure data quality. First, trade balance conditions are verified: $\sum_{i,k} X_{nik} - \sum_{i,k} \pi_{ink} X_{ik} = 0$ for each country. Second, expenditure shares sum to unity: $\sum_{k} \alpha_{nk} = 1$ for all countries. Third, factor and intermediate shares are consistent with gross output: $\beta_{ik} + \sum_{k'} \gamma_{ikk'} = 1$ for all country-sector pairs.

Missing tariff data affects less than 3\% of trade flows and is handled through sector-specific imputation based on Most Favored Nation (MFN) rates and regional trade agreement status. For the contemporary HTS dataset, we require at least 6 months of observations for each country-sector pair to ensure reliable tariff rate calculation, resulting in coverage of approximately 95\% of US import value.

Data quality is further validated through comparison with alternative sources. Our computed trade shares correlate at 0.94 with UN Comtrade bilateral flows, and our effective tariff rates match WITS ad valorem equivalents within 0.8 percentage points on average. These consistency checks confirm that our data integration preserves the underlying economic relationships while enabling comprehensive counterfactual analysis.

Table \ref{tab:data_coverage} summarizes the final dataset dimensions and coverage, while Table \ref{tab:tariff_scenarios} presents descriptive statistics for our four temporal tariff scenarios, highlighting the evolution of US trade policy over the 2024-2025 period.
