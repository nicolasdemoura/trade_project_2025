\section{Introduction}

\paragraph{} International trade theory has long grappled with fundamental questions about comparative advantage, specialization, and the welfare consequences of trade policy. The seminal work of \cite{eaton2002Econometrica} established a quantitative framework for analyzing trade flows through the lens of Ricardian comparative advantage, emphasizing the role of productivity differences and geographic barriers in shaping international trade patterns. Building on this foundation, \cite{costinot2012TheReviewofEconomicStudies} developed a multi-sector extension that provides deeper insights into the sectoral composition of trade and the mechanisms through which countries specialize according to their comparative advantages.

\paragraph{} The Costinot, Donaldson, and Komunjer (CDK) model represents a significant advancement in quantitative trade theory by incorporating multiple sectors, intermediate input linkages, and flexible production structures while maintaining the analytical tractability of the Ricardian framework. Unlike single-sector models, the CDK framework captures the complex interdependencies between sectors through input-output linkages, allowing for more realistic analysis of how trade policies propagate throughout an economy. This multi-sectoral structure is particularly important for understanding the distributional consequences of trade policy, as different sectors may be affected differently by changes in trade costs or tariffs.

\paragraph{} The empirical implementation of such quantitative trade models has become increasingly important for policy analysis, particularly in the context of recent trade policy developments. The ability to conduct counterfactual analysis, examining how changes in trade policies might affect welfare, trade flows, and sectoral outcomes, provides valuable insights for policymakers. However, the implementation of these models requires careful attention to parameter estimation, equilibrium computation, and the handling of multiple equilibrium conditions simultaneously.

\paragraph{} This paper contributes to the quantitative trade literature by providing a comprehensive empirical implementation of the CDK model with several methodological innovations. First, we develop a robust parameter estimation strategy using method of moments that jointly estimates iceberg trade costs, wages, and prices while ensuring strict adherence to trade balance conditions. Second, we implement a decomposed structure for iceberg trade costs that reduces the parameter space while maintaining economic interpretability. Third, we distinguish between mobile and immobile labor scenarios, interpreting these as short-run and long-run equilibria respectively. Finally, we conduct counterfactual analysis using real-world tariff data from multiple time periods, providing insights into the dynamic effects of trade policy changes.

\paragraph{} Our empirical application focuses on the welfare effects of recent U.S. tariff policies, using detailed trade data to construct multiple tariff scenarios representing different temporal aggregations (year-to-date, last twelve months, and quarterly averages). This approach allows us to examine not only the average effects of tariff changes but also their temporal variation and the role of seasonality in trade policy impacts. The analysis covers nine major countries/regions and nine broad economic sectors, providing a comprehensive view of how tariff changes propagate through the global economy.
