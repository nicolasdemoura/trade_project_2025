\section{Data Construction}

This section describes the comprehensive data construction process for our multi-sector Ricardian analysis. We combine multiple data sources to construct a consistent dataset covering 10 countries and 12 sectors, with particular attention to tariff scenarios for 2024-2025 policy analysis. Table \ref{tab:sector_mapping} presents our sector aggregation scheme linking WIOD classifications to HS codes, while Table \ref{tab:country_aggregation} details the country grouping structure.

\subsection{World Input-Output Database (WIOD)}

Our analysis builds on the 2016 release of the World Input-Output Database (WIOD) \citep{stehrer2014wiod}, providing comprehensive input-output tables for 43 countries and 56 sectors from 1995 to 2014. Following the literature standard, we focus on the year 2009 to avoid distortions from the global financial crisis while ensuring sufficient data coverage. The WIOD provides three critical data components: bilateral trade flows $X_{nik}$, intermediate input coefficients $\gamma_{ikk'}$, and final consumption patterns $\alpha_{nk}$ used to calibrate expenditure shares $\alpha_{nk}$ and intermediate input shares $\gamma_{ikk'}$ in our structural model.

We implement a structured aggregation scheme to balance tractability with economic realism. Our 10-country framework includes 8 focus economies (USA, Brazil, China, Japan, Mexico, India, Canada, United Kingdom), the European Union as an integrated bloc (27 member countries), and Rest of World capturing remaining economies. This aggregation captures the primary trade relationships while maintaining computational feasibility, as detailed in Table \ref{tab:country_aggregation}. The 56 original WIOD sectors are aggregated into 12 economically meaningful categories following the mapping scheme presented in Table \ref{tab:sector_mapping}: Food, Textiles, Paper, Chemical, Metal, Manufacture, Mining, Energy, Construction, Retail/Wholesale, Transport, and Services.

The aggregation preserves key economic relationships by maintaining consistent input-output coefficients and ensuring that trade balance conditions hold at the aggregated level. We verify that $\sum_{i} \sum_{k} X_{nik} = \sum_{i} \sum_{k} \pi_{nik} X_{nk}$ for all countries $n$, where $\pi_{nik}$ represents the trade share of country $n$'s imports from country $i$ in sector $k$.

\subsection{Socioeconomic Accounts and Labor Data}

Labor market data comes from the WIOD Socioeconomic Accounts (SEA) \citep{stehrer2014wiod}, providing employment $L_{nk}$ and compensation data by country and sector for 2009. We use total hours worked rather than employment counts to capture differences in work intensity across countries and sectors. This choice is particularly important for services sectors where part-time employment varies significantly across countries.

Exchange rate adjustments use annual average rates from the IMF International Financial Statistics to convert all monetary values to 2009 US dollars. This ensures comparability across countries while avoiding short-term volatility that might distort structural parameters. Labor shares $\beta_{ik}$ are computed from the WIOD input-output tables as the ratio of labor compensation to gross output: $\beta_{ik} = \frac{\text{Labor Compensation}_{ik}}{\text{Gross Output}_{ik}}$, ensuring consistency with the Cobb-Douglas production function specification in equation (\ref{eq:production}). Similarly, intermediate input coefficients $\gamma_{ikk'}$ are derived as $\gamma_{ikk'} = \frac{\text{Intermediate Purchases}_{ikk'}}{\text{Gross Output}_{ik}}$, where purchases of sector $k'$ inputs by sector $k$ in country $i$ are obtained directly from the WIOD input-output matrices.

\subsection{Tariff Data Architecture}

Our empirical analysis employs a dual-source approach to capture both historical patterns and contemporary policy dynamics. The baseline tariff structure uses the TRAINS database through WITS, providing ad valorem equivalent rates for 2009 that match our WIOD reference year. These historical tariffs establish the structural relationships and serve as the counterfactual baseline.

For policy-relevant analysis, we integrate a comprehensive HTS-level dataset covering US imports for 2024-2025. This dataset contains monthly observations of General Customs Value and General Import Charges at the 10-digit HTS level, allowing us to compute effective tariff rates as $\tau_{nik} = \frac{\text{Import Charges}}{\text{Customs Value}}$ for each trading partner and product category.

The HTS data enables construction of four temporal tariff scenarios: (1) \texttt{tariff\_rate24} uses 2024 annual averages as the baseline, (2) \texttt{tariff\_rate25\_YTD} covers year-to-date 2025 through the most recent month, (3) \texttt{tariff\_rate25\_LTM} uses a rolling 12-month window ending with the latest data, and (4) \texttt{tariff\_rate25\_3M} captures the most recent quarterly trends. This temporal structure allows analysis of evolving trade policies and seasonal patterns.

HTS codes are mapped to our 12-sector classification using concordance tables based on 2-digit HS categories. For example, HS chapters 28-39 map to Chemicals, chapters 84-85 to Energy/Machinery, and chapters 01-24 to Food products. Country mapping aggregates individual trading partners into our three-region framework based on trade volumes and economic integration patterns. Trade-weighted averaging ensures that tariff rates reflect the economic importance of different products and partners within each sector.

\subsection{Data Integration and Validation}

The final dataset integrates all sources into a consistent framework covering countries $n \in \{$USA, EU, RoW$\}$ and sectors $k \in \{1, ..., 12\}$. We implement several validation procedures to ensure data quality. First, trade balance conditions are verified: $\sum_{i,k} X_{nik} - \sum_{i,k} \pi_{ink} X_{ik} = 0$ for each country. Second, expenditure shares sum to unity: $\sum_{k} \alpha_{nk} = 1$ for all countries. Third, factor and intermediate shares are consistent with gross output: $\beta_{ik} + \sum_{k'} \gamma_{ikk'} = 1$ for all country-sector pairs.

Missing tariff data affects less than 3\% of trade flows and is handled through sector-specific imputation based on Most Favored Nation (MFN) rates and regional trade agreement status. For the contemporary HTS dataset, we require at least 6 months of observations for each country-sector pair to ensure reliable tariff rate calculation, resulting in coverage of approximately 95\% of US import value.

Data quality is further validated through comparison with alternative sources. Our computed trade shares correlate at 0.94 with UN Comtrade bilateral flows, and our effective tariff rates match WITS ad valorem equivalents within 0.8 percentage points on average. These consistency checks confirm that our data integration preserves the underlying economic relationships while enabling comprehensive counterfactual analysis.

Table \ref{tab:data_coverage} summarizes the final dataset dimensions and coverage, while Table \ref{tab:tariff_scenarios} presents descriptive statistics for our four temporal tariff scenarios, highlighting the evolution of US trade policy over the 2024-2025 period.

\begin{table}[H]
\centering
\caption{Sector Mapping: Model Sectors to WIOD and HS Classifications (Part 1)}
\label{tab:sector_mapping_1}
\footnotesize
\renewcommand{\arraystretch}{1.2}
\resizebox{\textwidth}{!}{%
\begin{tabular}{>{\raggedright}p{2.5cm} >{\raggedright}p{4cm} >{\raggedright\arraybackslash}p{10cm}}
\toprule
\multirow{1}{2.5cm}{\textbf{Model Sector}} & \multirow{1}{4cm}{\textbf{WIOD Sectors}} & \multirow{1}{10cm}{\textbf{HS Codes (Chapters)}} \\
\midrule
\multirow{5}{2.5cm}{\textbf{Chemical}} & \multirow{5}{4cm}{Chemicals and Chemical Products (24, c9); Rubber and Plastics (25, c10)} & \multirow{5}{10cm}{28: Inorganic chemicals; 29: Organic chemicals; 30: Pharmaceutical products; 31: Fertilizers; 32: Tanning/dyeing extracts; 33: Essential oils/perfumery; 34: Soap/surface-active agents; 35: Albuminoidal substances; 36: Explosives/pyrotechnics; 38: Miscellaneous chemicals; 39: Plastics} \\
& & \\
& & \\
& & \\
& & \\
\midrule
\multirow{3}{2.5cm}{\textbf{Construction}} & \multirow{3}{4cm}{Construction (F, c18); Real Estate Activities (70, c29)} & \multirow{3}{10cm}{25: Salt/sulphur/earths and stone; 68: Articles of stone/plaster/cement} \\
& & \\
& & \\
\midrule
\multirow{4}{2.5cm}{\textbf{Energy}} & \multirow{4}{4cm}{Coke, Refined Petroleum and Nuclear Fuel (23, c8); Electricity, Gas and Water Supply (E, c17)} & \multirow{4}{10cm}{27: Mineral fuels/oils; 84: Nuclear reactors/boilers/machinery} \\
& & \\
& & \\
& & \\
\midrule
\multirow{6}{2.5cm}{\textbf{Food}} & \multirow{6}{4cm}{Agriculture, Hunting, Forestry and Fishing (AtB, c1); Food, Beverages and Tobacco (15t16, c3)} & \multirow{6}{10cm}{1: Live animals; 2: Meat; 3: Fish/crustaceans; 4: Dairy produce; 5: Animal products; 6: Live trees/plants; 7: Vegetables; 8: Fruit/nuts; 9: Coffee/tea/spices; 10: Cereals; 11: Milling products; 12: Oil seeds; 15: Fats/oils; 16: Meat preparations; 17: Sugars; 18: Cocoa; 19: Cereal preparations; 20: Vegetable preparations; 21: Miscellaneous edible; 22: Beverages; 23: Food waste/animal feed; 24: Tobacco} \\
& & \\
& & \\
& & \\
& & \\
& & \\
\midrule
\multirow{9}{2.5cm}{\textbf{Manufacture}} & \multirow{9}{4cm}{Electrical and Optical Equipment (30t33, c14); Machinery, Nec (29, c13); Manufacturing, Nec; Recycling (36t37, c16); Transport Equipment (34t35, c15)} & \multirow{9}{10cm}{37: Photographic goods; 40: Rubber articles; 41: Raw hides/skins; 42: Leather articles; 43: Furskins; 45: Cork articles; 46: Straw manufactures; 64: Footwear; 65: Headgear; 66: Umbrellas; 67: Feathers; 69: Ceramics; 70: Glass; 71: Precious stones; 82: Tools/cutlery; 83: Miscellaneous base metal; 85: Electrical machinery; 86: Railway vehicles; 87: Motor vehicles; 88: Aircraft; 89: Ships; 90: Optical instruments; 91: Clocks/watches; 92: Musical instruments; 93: Arms/ammunition; 94: Furniture; 95: Toys/games; 96: Miscellaneous manufactures; 97: Art/antiques} \\
& & \\
& & \\
& & \\
& & \\
& & \\
& & \\
& & \\
& & \\
\bottomrule
\end{tabular}%
}
\begin{tablenotes}
\footnotesize
\item Notes: WIOD sector codes in parentheses show both the alphanumeric (first) and numeric (second) identifiers from \cite{timmer2015illustrated}. HS codes refer to 2-digit Harmonized System chapters. Table continues on next page.
\end{tablenotes}
\end{table}

\begin{table}[H]
\centering
\caption{Sector Mapping: Model Sectors to WIOD and HS Classifications (Part 2)}
\label{tab:sector_mapping_2}
\footnotesize
\renewcommand{\arraystretch}{1.2}
\resizebox{\textwidth}{!}{%
\begin{tabular}{>{\raggedright}p{2.5cm} >{\raggedright}p{4cm} >{\raggedright\arraybackslash}p{10cm}}
\toprule
\multirow{1}{2.5cm}{\textbf{Model Sector}} & \multirow{1}{4cm}{\textbf{WIOD Sectors}} & \multirow{1}{10cm}{\textbf{HS Codes (Chapters)}} \\
\midrule
\multirow{4}{2.5cm}{\textbf{Metal}} & \multirow{4}{4cm}{Basic Metals and Fabricated Metal (27t28, c12); Other Non-Metallic Mineral (26, c11)} & \multirow{4}{10cm}{72: Iron/steel; 73: Iron/steel articles; 74: Copper; 75: Nickel; 76: Aluminium; 78: Lead; 80: Tin; 81: Other base metals; 79: Zinc} \\
& & \\
& & \\
& & \\
\midrule
\multirow{2}{2.5cm}{\textbf{Mining}} & \multirow{2}{4cm}{Mining and Quarrying (C, c2)} & \multirow{2}{10cm}{26: Ores/slag/ash; 13: Lac/gums/resins; 14: Vegetable plaiting materials} \\
& & \\
\midrule
\multirow{4}{2.5cm}{\textbf{Paper}} & \multirow{4}{4cm}{Pulp, Paper, Printing and Publishing (21t22, c7); Wood and Products of Wood and Cork (20, c6)} & \multirow{4}{10cm}{44: Wood/wood articles; 47: Wood pulp; 48: Paper/paperboard} \\
& & \\
& & \\
& & \\
\midrule
\multirow{5}{2.5cm}{\textbf{Retail and Wholesale}} & \multirow{5}{4cm}{Retail Trade, Except Motor Vehicles (52, c21); Sale/Maintenance of Motor Vehicles (50, c19); Wholesale Trade (51, c20)} & \multirow{5}{10cm}{\textit{Non-tradable services sector}} \\
& & \\
& & \\
& & \\
& & \\
\midrule
\multirow{13}{2.5cm}{\textbf{Services}} & \multirow{13}{4cm}{Education (M, c32); Financial Intermediation (J, c28); Health and Social Work (N, c33); Hotels and Restaurants (H, c22); Other Community/Social/Personal Services (O, c34); Post and Telecommunications (64, c27); Private Households with Employed Persons (P, c35); Public Admin and Defence (L, c31); Renting of M\&Eq and Other Business Activities (71t74, c30)} & \multirow{13}{10cm}{\textit{Non-tradable services sector}} \\
& & \\
& & \\
& & \\
& & \\
& & \\
& & \\
& & \\
& & \\
& & \\
& & \\
& & \\
& & \\
\midrule
\multirow{4}{2.5cm}{\textbf{Textiles}} & \multirow{4}{4cm}{Leather, Leather and Footwear (19, c5); Textiles and Textile Products (17t18, c4)} & \multirow{4}{10cm}{50: Silk; 51: Wool/animal hair; 52: Cotton; 53: Other vegetable fibers; 54: Man-made filaments; 55: Man-made staple fibers; 56: Wadding/felt; 57: Carpets; 58: Special woven fabrics; 59: Impregnated textiles; 60: Knitted fabrics; 61: Knitted apparel; 62: Woven apparel; 63: Other textiles} \\
& & \\
& & \\
& & \\
\midrule
\multirow{5}{2.5cm}{\textbf{Transport}} & \multirow{5}{4cm}{Air Transport (62, c25); Inland Transport (60, c23); Other Supporting Transport Activities (63, c26); Water Transport (61, c24)} & \multirow{5}{10cm}{\textit{Non-tradable services sector}} \\
& & \\
& & \\
& & \\
& & \\
\bottomrule
\end{tabular}%
}
\begin{tablenotes}
\footnotesize
\item Notes: WIOD sector codes in parentheses show both the alphanumeric (first) and numeric (second) identifiers from \cite{timmer2015illustrated}. HS codes refer to 2-digit Harmonized System chapters. Non-tradable services sectors do not have corresponding HS classifications as they represent domestic activities.
\end{tablenotes}
\end{table}

\begin{table}[H]
\centering
\caption{Country Aggregation Scheme}
\label{tab:country_aggregation}
\renewcommand{\arraystretch}{1.3}
\resizebox{0.9\textwidth}{!}{%
\begin{tabular}{>{\raggedright}p{3cm} >{\raggedright}p{8cm} >{\raggedright\arraybackslash}p{4cm}}
\toprule
\textbf{Model Region} & \textbf{Constituent Countries/Economies} & \textbf{WIOD Codes} \\
\midrule
\multirow{8}{3cm}{\textbf{Focus Economies}} & United States & USA \\
& Brazil & BRA \\
& China & CHN \\
& Japan & JPN \\
& Mexico & MEX \\
& India & IND \\
& Canada & CAN \\
& United Kingdom & GBR \\
\midrule
\multirow{6}{3cm}{\textbf{European Union}} & \multirow{6}{8cm}{Austria, Belgium, Bulgaria, Croatia, Cyprus, Czech Republic, Denmark, Estonia, Finland, France, Germany, Greece, Hungary, Ireland, Italy, Latvia, Lithuania, Luxembourg, Malta, Netherlands, Poland, Portugal, Romania, Slovakia, Slovenia, Spain, Sweden} & \multirow{6}{4cm}{AUT, BEL, BGR, HRV, CYP, CZE, DNK, EST, FIN, FRA, DEU, GRC, HUN, IRL, ITA, LVA, LTU, LUX, MLT, NLD, POL, PRT, ROU, SVK, SVN, ESP, SWE} \\
& & \\
& & \\
& & \\
& & \\
& & \\
\midrule
\multirow{3}{3cm}{\textbf{Rest of World}} & \multirow{3}{8cm}{Australia, Indonesia, Korea, Russia, Turkey, Taiwan, Other economies} & \multirow{3}{4cm}{AUS, IDN, KOR, RUS, TUR, TWN, RoW} \\
& & \\
& & \\
\bottomrule
\end{tabular}%
}
\begin{tablenotes}
\footnotesize
\item Notes: The aggregation scheme balances analytical tractability with economic realism. Focus economies represent the 8 largest trading partners and policy-relevant countries for our analysis. The European Union is treated as an integrated economic area reflecting deep trade and regulatory integration among member states. Rest of World captures remaining economies in the WIOD database. Country codes follow ISO 3166-1 alpha-3 standard as implemented in \cite{timmer2015illustrated}.
\end{tablenotes}
\end{table}
