\section{Data Construction}

This section presents the comprehensive data construction process underlying our multi-sector Ricardian analysis. We integrate multiple international databases to build a consistent analytical framework covering 10 countries and 12 sectors, with particular focus on constructing time-varying tariff scenarios for contemporary US trade policy evaluation during 2024-2025. Our data architecture combines trade data from 2009 with current policy instruments to enable robust counterfactual analysis. Tables \ref{tab:sector_mapping_1} and \ref{tab:sector_mapping_2} present our sector aggregation scheme linking WIOD classifications to HS codes, while Table \ref{tab:country_aggregation} details the country grouping structure.

\subsection{World Input-Output Database (WIOD)}

Our analysis builds on the World Input-Output Database (WIOD) 2013 Release \citep{timmer2015illustrated}, providing comprehensive input-output tables for 27 EU countries and 13 other major economies from 1995 to 2011. We focus on the year 2009 because it offers the most comprehensive trade flow information across our country sample, which is essential for robust estimation of the structural parameters. This dataset includes detailed information on bilateral intermediate input flows $Z_{nikl}$ representing imports by sector $k$ in country $n$ from sector $l$ in country $i$. This data allow us to compute three critical data components that form the foundation of our calibration: bilateral trade flows $X_{nik} = \sum_{l} Z_{nikl}$ used to compute trade shares $\pi_{nik}$, sectoral intermediate input coefficients $\gamma_{ikk'}$, and final expenditure patterns used to derive consumption shares $\alpha_{nk}$.

We implement a structured aggregation scheme to balance tractability with economic realism. Our 10-country framework includes 8 focus economies (USA, Brazil, China, Japan, Mexico, India, Canada, United Kingdom), the European Union as an integrated bloc (27 member countries), and Rest of World capturing remaining economies. This aggregation captures the primary trade relationships while maintaining computational feasibility, as detailed in Table \ref{tab:country_aggregation}. 

The 35 original WIOD sectors are aggregated into 12 economically meaningful categories following the mapping scheme presented in Tables \ref{tab:sector_mapping_1} and \ref{tab:sector_mapping_2}: Food, Textiles, Paper, Chemical, Metal, Manufacture, Mining, Energy, Construction, Retail/Wholesale, Transport, and Services. This aggregation scheme is designed to capture key sectoral distinctions while maintaining sufficient observations within each category to ensure robust parameter estimation. We also match these sectors to corresponding HS codes to facilitate integration with tariff data from TRAINS and USITC, as detailed in the sector mapping tables.

\input{Tables/country_aggregation}

\input{Tables/sector_mapping}

\subsection{Socioeconomic Accounts and Labor Data}

Labor market data comes from the WIOD Socioeconomic Accounts (SEA) July 2014 release \citep{timmer2015illustrated}, providing employment and compensation data by country and sector for 2009. 

All monetary values are converted to 2009 US dollars using annual average exchange rates from the IMF International Financial Statistics, ensuring cross-country comparability. The WIOD input-output framework enables direct computation of key structural parameters: labor shares $\beta_{ik} = \frac{\text{Labor Compensation}_{ik}}{\text{Gross Output}_{ik}}$ capture the factor intensity of production consistent with our Cobb-Douglas specification. Together with intermediate input coefficients $\gamma_{ikk'} = \frac{\text{Intermediate Purchases}_{ikk'}}{\text{Gross Output}_{ik}}$ and final expenditure shares $\alpha_{nk} = \frac{\text{Final Consumption}_{nk}}{\text{Total Final Consumption}_{n}}$, these parameters form the technological and preference foundations of our structural model.

\subsection{Tariff Data Architecture}

Our empirical analysis employs a dual-source approach to capture both comprehensive international coverage and high-frequency US policy dynamics. We integrate tariff data from the TRAINS database for international comparability with detailed USITC customs data for contemporary US trade policy analysis, enabling robust counterfactual evaluation of recent policy changes while maintaining structural consistency across our multi-country framework.

The TRAINS database, accessed through the World Bank's WITS platform \citep{WITS2025}, provides our baseline tariff structure for all non-US economies in the sample. We utilize 2023 data as the most recent year with comprehensive coverage across our 9 international partners (Brazil, China, Japan, Mexico, India, Canada, United Kingdom, European Union, and Rest of World). The TRAINS system offers ad valorem equivalent rates at the HS 6-digit level, which we aggregate to match our 12-sector classification scheme. This database ensures consistent measurement methodology across countries and provides the structural foundation for calibrating bilateral trade costs in our baseline equilibrium.

For the United States, we employ detailed HTS-level data from USITC DataWeb covering 2024-2025 \citep{USITC2025}. This high-frequency dataset contains monthly observations of General Customs Value and General Import Charges at the 10-digit HTS level, enabling precise calculation of effective tariff rates as $\tau_{ikt} = \frac{\text{Import Charges}_{ikt}}{\text{Customs Value}_{ikt}}$ for each trading partner and product category. The USITC data captures recent policy implementations, including antidumping duties, countervailing duties, and Section 301 tariffs, providing the granular information necessary for evaluating contemporary trade policy scenarios. In order to simplify, we consider this implied tariff as the total tariff faced by US importers, ignoring potential interactions with other trade barriers.

We calibrate the structural model using 2023 tariff data as our baseline equilibrium for international partners, combined with 2024 US rates as the initial state for counterfactual analysis. For policy evaluation, we construct three temporal aggregations of 2025 US tariff policies: a rolling 12-month window to capture medium-term policy trends, year-to-date averages through the most recent available month, and the latest quarterly period to identify short-term policy shifts. While the quarterly (3-month) aggregation provides the most accurate representation of current policy stance, it suffers from seasonal fluctuations in trade patterns that can distort effective tariff calculations. Extending to a 12-month rolling window mitigates these seasonal effects but incorporates pre-2025 tariff policies that may not reflect the current administration's trade stance. This approach maintains international comparability through TRAINS while leveraging USITC's superior temporal resolution for US policy analysis, with the quarterly measure serving as our primary specification despite its seasonal limitations.
