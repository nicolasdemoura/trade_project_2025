\begin{table}[H]
\centering
\caption{Sector Mapping: Model Sectors to WIOD and HS Classifications (Part 1)}
\label{tab:sector_mapping_1}
\footnotesize
\renewcommand{\arraystretch}{1.2}
\resizebox{\textwidth}{!}{%
\begin{tabular}{>{\raggedright}p{2.5cm} >{\raggedright}p{4cm} >{\raggedright\arraybackslash}p{10cm}}
\toprule
\textbf{Model Sector} & \textbf{WIOD Sectors} & \textbf{HS Codes (Chapters)} \\
\midrule
\multirow{4}{2.5cm}{\textbf{Chemical}} & Chemicals and Chemical Products (24, c9) & 28: Inorganic chemicals; 29: Organic chemicals; 30: Pharmaceutical products; \\
& Rubber and Plastics (25, c10) & 31: Fertilizers; 32: Tanning/dyeing extracts; 33: Essential oils/perfumery; \\
& & 34: Soap/surface-active agents; 35: Albuminoidal substances; \\
& & 36: Explosives/pyrotechnics; 38: Miscellaneous chemicals; 39: Plastics \\
\midrule
\multirow{2}{2.5cm}{\textbf{Construction}} & Construction (F, c18) & \multirow{2}{10cm}{25: Salt/sulphur/earths and stone; 68: Articles of stone/plaster/cement} \\
& Real Estate Activities (70, c29) & \\
\midrule
\multirow{2}{2.5cm}{\textbf{Energy}} & Coke, Refined Petroleum and Nuclear Fuel (23, c8) & \multirow{2}{10cm}{27: Mineral fuels/oils; 84: Nuclear reactors/boilers/machinery} \\
& Electricity, Gas and Water Supply (E, c17) & \\
\midrule
\multirow{6}{2.5cm}{\textbf{Food}} & Agriculture, Hunting, Forestry and Fishing (AtB, c1) & 1: Live animals; 2: Meat; 3: Fish/crustaceans; 4: Dairy produce; \\
& Food, Beverages and Tobacco (15t16, c3) & 5: Animal products; 6: Live trees/plants; 7: Vegetables; 8: Fruit/nuts; \\
& & 9: Coffee/tea/spices; 10: Cereals; 11: Milling products; 12: Oil seeds; \\
& & 15: Fats/oils; 16: Meat preparations; 17: Sugars; 18: Cocoa; \\
& & 19: Cereal preparations; 20: Vegetable preparations; 21: Miscellaneous edible; \\
& & 22: Beverages; 23: Food waste/animal feed; 24: Tobacco \\
\midrule
\multirow{8}{2.5cm}{\textbf{Manufacture}} & Electrical and Optical Equipment (30t33, c14) & 37: Photographic goods; 40: Rubber articles; 41: Raw hides/skins; \\
& Machinery, Nec (29, c13) & 42: Leather articles; 43: Furskins; 45: Cork articles; 46: Straw manufactures; \\
& Manufacturing, Nec; Recycling (36t37, c16) & 64: Footwear; 65: Headgear; 66: Umbrellas; 67: Feathers; \\
& Transport Equipment (34t35, c15) & 69: Ceramics; 70: Glass; 71: Precious stones; 82: Tools/cutlery; \\
& & 83: Miscellaneous base metal; 85: Electrical machinery; 86: Railway vehicles; \\
& & 87: Motor vehicles; 88: Aircraft; 89: Ships; 90: Optical instruments; \\
& & 91: Clocks/watches; 92: Musical instruments; 93: Arms/ammunition; \\
& & 94: Furniture; 95: Toys/games; 96: Miscellaneous manufactures; 97: Art/antiques \\
\midrule
\multirow{2}{2.5cm}{\textbf{Metal}} & Basic Metals and Fabricated Metal (27t28, c12) & 72: Iron/steel; 73: Iron/steel articles; 74: Copper; 75: Nickel; \\
& Other Non-Metallic Mineral (26, c11) & 76: Aluminium; 78: Lead; 80: Tin; 81: Other base metals; 79: Zinc \\
\bottomrule
\end{tabular}%
}
\begin{tablenotes}
\footnotesize
\item Notes: WIOD sector codes in parentheses show both the alphanumeric (first) and numeric (second) identifiers from \cite{stehrer2014wiod}. HS codes refer to 2-digit Harmonized System chapters. Table continues on next page.
\end{tablenotes}
\end{table}

\begin{table}[H]
\centering
\caption{Sector Mapping: Model Sectors to WIOD and HS Classifications (Part 2)}
\label{tab:sector_mapping_2}
\footnotesize
\renewcommand{\arraystretch}{1.2}
\resizebox{\textwidth}{!}{%
\begin{tabular}{>{\raggedright}p{2.5cm} >{\raggedright}p{4cm} >{\raggedright\arraybackslash}p{10cm}}
\toprule
\textbf{Model Sector} & \textbf{WIOD Sectors} & \textbf{HS Codes (Chapters)} \\
\midrule
\textbf{Mining} & Mining and Quarrying (C, c2) & 26: Ores/slag/ash; 13: Lac/gums/resins; 14: Vegetable plaiting materials \\
\midrule
\multirow{2}{2.5cm}{\textbf{Paper}} & Pulp, Paper, Printing and Publishing (21t22, c7) & \multirow{2}{10cm}{44: Wood/wood articles; 47: Wood pulp; 48: Paper/paperboard} \\
& Wood and Products of Wood and Cork (20, c6) & \\
\midrule
\multirow{3}{2.5cm}{\textbf{Retail and Wholesale}} & Retail Trade, Except Motor Vehicles (52, c21) & \multirow{3}{10cm}{\textit{Non-tradable services sector}} \\
& Sale/Maintenance of Motor Vehicles (50, c19) & \\
& Wholesale Trade and Commission Trade (51, c20) & \\
\midrule
\multirow{8}{2.5cm}{\textbf{Services}} & Education (M, c32) & \multirow{8}{10cm}{\textit{Non-tradable services sector}} \\
& Financial Intermediation (J, c28) & \\
& Health and Social Work (N, c33) & \\
& Hotels and Restaurants (H, c22) & \\
& Other Community/Social/Personal Services (O, c34) & \\
& Post and Telecommunications (64, c27) & \\
& Private Households with Employed Persons (P, c35) & \\
& Public Admin and Defence (L, c31) & \\
& Renting of M\&Eq and Other Business Activities (71t74, c30) & \\
\midrule
\multirow{4}{2.5cm}{\textbf{Textiles}} & Leather, Leather and Footwear (19, c5) & 50: Silk; 51: Wool/animal hair; 52: Cotton; 53: Other vegetable fibers; \\
& Textiles and Textile Products (17t18, c4) & 54: Man-made filaments; 55: Man-made staple fibers; 56: Wadding/felt; \\
& & 57: Carpets; 58: Special woven fabrics; 59: Impregnated textiles; \\
& & 60: Knitted fabrics; 61: Knitted apparel; 62: Woven apparel; 63: Other textiles \\
\midrule
\multirow{4}{2.5cm}{\textbf{Transport}} & Air Transport (62, c25) & \multirow{4}{10cm}{\textit{Non-tradable services sector}} \\
& Inland Transport (60, c23) & \\
& Other Supporting Transport Activities (63, c26) & \\
& Water Transport (61, c24) & \\
\bottomrule
\end{tabular}%
}
\begin{tablenotes}
\footnotesize
\item Notes: WIOD sector codes in parentheses show both the alphanumeric (first) and numeric (second) identifiers from \cite{stehrer2014wiod}. HS codes refer to 2-digit Harmonized System chapters. Non-tradable services sectors do not have corresponding HS classifications as they represent domestic activities.
\end{tablenotes}
\end{table}
