\begin{table}[H]
\centering
\caption{Sector Mapping: Model Sectors to WIOD and HS Classifications (Part 1)}
\label{tab:sector_mapping_1}
\footnotesize
\renewcommand{\arraystretch}{1.2}
\resizebox{\textwidth}{!}{%
\begin{tabular}{>{\raggedright}p{2.5cm} >{\raggedright\arraybackslash}m{14cm}}
\toprule
\textbf{Model Sector} & \textbf{WIOD Sectors and HS Codes (Chapters)} \\
\midrule
\multirow{4}{2.5cm}{\textbf{Chemical}} & \textbf{WIOD:} Chemicals and Chemical Products (24, c9); Rubber and Plastics (25, c10) \\
& \textbf{HS:} 28: Inorganic chemicals; 29: Organic chemicals; 30: Pharmaceutical products; 31: Fertilizers; 32: Tanning/dyeing extracts; 33: Essential oils/perfumery; 34: Soap/surface-active agents; 35: Albuminoidal substances; 36: Explosives/pyrotechnics; 38: Miscellaneous chemicals; 39: Plastics \\
\midrule
\multirow{2}{2.5cm}{\textbf{Construction}} & \textbf{WIOD:} Construction (F, c18); Real Estate Activities (70, c29) \\
& \textbf{HS:} 25: Salt/sulphur/earths and stone; 68: Articles of stone/plaster/cement \\
\midrule
\multirow{2}{2.5cm}{\textbf{Energy}} & \textbf{WIOD:} Coke, Refined Petroleum and Nuclear Fuel (23, c8); Electricity, Gas and Water Supply (E, c17) \\
& \textbf{HS:} 27: Mineral fuels/oils; 84: Nuclear reactors/boilers/machinery \\
\midrule
\multirow{6}{2.5cm}{\textbf{Food}} & \textbf{WIOD:} Agriculture, Hunting, Forestry and Fishing (AtB, c1); Food, Beverages and Tobacco (15t16, c3) \\
& \textbf{HS:} 1: Live animals; 2: Meat; 3: Fish/crustaceans; 4: Dairy produce; 5: Animal products; 6: Live trees/plants; 7: Vegetables; 8: Fruit/nuts; 9: Coffee/tea/spices; 10: Cereals; 11: Milling products; 12: Oil seeds; 15: Fats/oils; 16: Meat preparations; 17: Sugars; 18: Cocoa; 19: Cereal preparations; 20: Vegetable preparations; 21: Miscellaneous edible; 22: Beverages; 23: Food waste/animal feed; 24: Tobacco \\
\bottomrule
\end{tabular}%
}
\begin{tablenotes}
\footnotesize
\item Notes: WIOD sector codes in parentheses show both the alphanumeric (first) and numeric (second) identifiers from \cite{stehrer2014wiod}. HS codes refer to 2-digit Harmonized System chapters.
\end{tablenotes}
\end{table}
